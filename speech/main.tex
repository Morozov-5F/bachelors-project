%!TEX TS-program = xelatex
%!TEX encoding = UTF-8 Unicode
\documentclass[a4paper,14pt]{extarticle}

\usepackage{fontspec, xltxtra, xunicode, polyglossia}
\setmainlanguage{russian}
\setotherlanguage{english}
\setkeys{russian}{babelshorthands=true}

\usepackage[bottom=2cm, top=1cm, left=1cm, right=1cm]{geometry}

\setmainfont{CMU Serif}
\setmonofont{CMU Typewriter Text}

\usepackage{titlesec}
\titleformat{\section}{\normalfont\Large\bfseries}{Слайд~\thesection:}{0.5em}{}{}

\setlength{\parindent}{0cm}

\usepackage{amssymb}

\begin{document}
\section{Приветствие}
Уважаемая Государственная экзаменационная комиссия, позвольте представить Вашему вниманию выпускную работу бакалавра. Я, Морозов Евгений Юрьевич, научный руководитель "--- доктор физико-математических наук, профессор кафедры Цифровых Технологий Лобода Александр Васильевич. Тема дипломной работы "--- <<Исследование симметрий алгебраических уравнений>>.
\\~\\
Симметрии различных математических объектов представляют в современной науке немалый интерес. В данной работе исследуются симметрии одного семейства кубических гиперповерхностей в трехмерном комплексном пространстве. Также работа тесно связана с изучение свойств однородности в этом пространстве. Задача является чисто математической, но многомерной и сложной, поэтому в ходе ее решния нельзя былло обойтись без компьютерных средств.


\section{Постановка задачи}
Рассмотрим класс <<квадро-кубических>> вещественных гиперповерхностей в $\mathbb{C}^3$ (формула 1 на слайде). \textit{комментарий к формуле}.
\\~\\
Далее, симметрией поверхности будем считать любое аффинное преобразование, сохраняющей данную поверхность. Таким образом, \textbf{цель работы} заключается в изучении групп аффинных преобразований, сохраняющих данные поверхности, и, в частности, размерностей этих групп. Теперь я кратко изложу метод решения.

\section{Метод решения}
Очевидно, что любая поверхность из данного семейства сохраняется сдвигом по переменной $u = \Re(w)$. Таким образом, размерность каждой из изучаемых групп преобразований не меньше 1. Для изучения групп больших размерностей естественно использовать однопараметричекие группы "--- могопараметрическая группа "--- это, по сути, совокупность независимых однопараметрических групп.
\\~\\
В таком случае, обозначим через $F_t$ однопараметрическую группу аффинных преобразований, сохранаяющих уравнение поверхности. Для удобства будем считать, что тождественное преобразование в этой группе соответствует нулевому значению параметра. Мы будем считать, что поверхность сохраянется под действием группы с точностью до некоторого ненулевого множителя.

\section{Метод решения: продолжение}
Далее продифференцируем уравнение на сохранение поверхности с предыдущего слайда по параметру $t$ в точке ноль и сузим результат на поверхность $M$. В результате данного действия можно перейти к однородной системе линейных алгебраических уравнений, ранг матрицы которой определяет искомую размерность группы преобразований.
\\~\\
Таким образом, исследование размерность группы аффинных преобразований сводится к исследованию ранга полученной матрицы.
\\~\\
Следует заметить, что особый интерес представляют поверхности, для которых размерность группы преобразований больше либо равна 5. Такие поверхности могут являться аффинно-однородными. Это означает, что под действиями преобразований их этой группы, можно сдвигаться вдоль этой поверхности в любом направлении.

\section{Частный случай}
В связи со сложностью задачи, изначально был рассмотрен частный случай семейства "--- это поверхности, в которых квадратичная форма $Q$ не зависит от переменных $x_2, y_2$ (общий вид такой поверхности можно видеть на слайде). Следует заметить, что за счет аффинных преобразований можно фактически свести количество параметров квадратичной формы до одного. Применяя описанный метод, получаем систему из 57 уравнений относительно 24 неизвестных. Далее приведем оценку размерностей группы для этого слуачя.

\section{Частный случай: оценка размерности}
Можно сформулировать эту оценку в виде теоремы 1: размерность группы Ли аффинных преобразований, сохраняющих любую поверхность из частного семейства изменяется от 3 до 5 и каждая из таких размерностей достижима. В теореме так же перечислены классы поверхностей, на которых достигаются различные размерности.
\\~\\
Отметим, что размерностям 5 соответствуют уже известные аффинно-однородные поверхности в $\mathbb{C}^3$.

\section{Частный случай: допустимые движения}
На данном слайде можно увидеть допустимые движения для каждой из резмерностей. Для любой поверхности из т.н. <<сокращенного>> семейства существуют три типа движенй "--- указанный сдвиг опо $u$, масштабирование и скользящий поворот.
\\~\\
Для поверхностей, с размерностью группы 4, добавляется еще одно движение "--- повороты в плоскости $z_1$.
\\~\\
Для поверхностей с размерностью группы 5, в дополнение к трем основным движения добавляются еще два "--- сдвиг по $y_1$ и поворот.

\section{Частный случай (еще одна теорема)}
Дополнительным результатом работы является следующая теорема, которая несколько расширяет класс уже известных аффинно-однородных поверхностей, представленный в статье Атанова Лободы и Шиповской. Расширение происходит за счет усложнения кубической части.
\\~\\
Перейдем теперь к общему случаю.

\section{Общий случай}
В общем случае возникает система из 83 уравнений отсносительно 24 неизвестных. Заметим, что сложность исследование заключается не только в размерах системы, но и в том, что коэффициенты при неизвестных в этой системе зависят от параметров квадратичной формы \textbf{полиномиально}.

\section{Общий случай (теорема)}
Основным результатом данной работы является следующая теорема:
\\~\\
В качестве примеров на слайде приведены г

\section{Заключение}
В данной работе были исследованы симметрии одного семейства кубических гиперповерхностей в пространстве $\mathbb{C}^3$. Основным результатом работы является оценка размерностей групп аффинных преобразований, сохраняющих поверхности из данного семейства. В ходе работы была написана программа на языке {\ttfamily Wolfram Language}для нахождения размерностей групп преобразований для конкретных поверхностей. Как показывает практика, из-за большого количества неизвестных параметров, задача не может бытть полностью решена компьютером, поэтому приходится проверять результаты и совершать некоторые действия вручную. Также в ходе работы был несколько расширен класс известных аффинно-однородных поверхностей.
Спасибо за внимание.

\clearpage
\titleformat{\section}{\normalfont\Large\bfseries}{Приложение~\thesection}{0.5em}{}{}
\appendix
\section{Термины}
\begin{itemize}
\item \textbf{Гиперповерхность} "--- обобщение понятия обычной поверхности трехмерного пространства на случай n-мерного пространства. Размерность гиперповерхности на единицу меньше размерности объемлющего пространства.

\item \textbf{Группа} "--- непустое множество, замкнутое относительное бинарной операции, если выполнены три аксиомы "--- ассоциативность, наличие нейтрального элемента, наличие обратного элемента. Группа является \textbf{абелевой}, если операция коммутативна.

\item \textbf{Размерность параметрической группы} "--- число парметров, от которых зависят элементы группы.

\item \textbf{Инфинитезимальное преобразование} "--- преобразование, устанавливающее в соответствие каждой точке $p$ поверхности $M$ вектор, касательный к $p$. Иными словами, оно показывает направление движения точки $p$ в результате действия группы $F_t$ для бесконечно малых $t$.

\end{itemize}

\end{document}