%!TEX TS-program = xelatex
%!TEX encoding = UTF-8 Unicode
\documentclass[a4paper,14pt]{extarticle}

\usepackage{fontspec, xltxtra, xunicode, polyglossia}
\setmainlanguage{russian}
\setotherlanguage{english}
\setkeys{russian}{babelshorthands=true}

\usepackage[bottom=2cm, top=1cm, left=1cm, right=1cm]{geometry}

\setmainfont{CMU Serif}

\usepackage{titlesec}
\titleformat{\section}{\normalfont\Large\bfseries}{Слайд~\thesection:}{0.5em}{}{}

\setlength{\parindent}{0cm}

\usepackage{amssymb}

\begin{document}
\section{Приветствие}
Уважаемая Государственная экзаменационная комиссия, позвольте представить Вашему вниманию выпускную работу бакалавра. Я, Морозов Евгений Юрьевич, научный руководитель "--- доктор физико-математических наук, профессор кафедры Цифровых Технологий Лобода Александр Васильевич. Тема дипломной работы "--- <<Исследование симметрий алгебраических уравнений>>.
\\~\\
Симметрии различных математических объектов представляют в современной науке немалый интерес. В данной работе исследуются симметрии одного семейства кубических гиперповерхностей в трехмерном комплексном пространстве. Также работа тесно связана с изучение свойств однородности в этом пространстве.


\section{Постановка задачи}
Рассмотрим класс <<квадро-кубических>> вещественных гиперповерхностей в $\mathbb{C}^3$ (формула 1 на слайде). \textit{комментарий к формуле}.
\\~\\
Далее, симметрией поверхности будем считать любое аффинное преобразование, сохраняющей данную поверхность. Таким образом, \textbf{цель работы} заключается в изучении групп аффинных преобразований, сохраняющих данные поверхности, и, в частности, размерностей этих групп. Теперь я кратко изложу метод решения.

\section{Метод решения}
Очевидно, что любая поверхность из данного семейства сохраняется сдвигом по переменной $u = \Re(w)$.

\section{Метод решения: продолжение}
~

\section{Частный случай}
~

\section{Частный случай: оценка размерности}
~

\section{Частный случай: допустимые движения}
~

\section{Частный случай (еще одна теорема)}
~

\section{Общий случай}
~

\section{Общий случай (теорема)}
~

\section{Заключение}
В данной работе были исследованы симметрии одного семейства кубических гиперповерхностей в пространстве $\mathbf{C}^3$. Основным результатом работы является оценка 

Спасибо за внимание.

\clearpage
\titleformat{\section}{\normalfont\Large\bfseries}{Приложение~\thesection}{0.5em}{}{}
\appendix
\section{Термины}
\begin{itemize}
\item \textbf{Гиперповерхность} "--- обобщение понятия обычной поверхности трехмерного пространства на случай n-мерного пространства. Размерность гиперповерхности на единицу меньше размерности объемлющего пространства.

\item \textbf{Группа} "--- непустое множество, замкнутое относительное бинарной операции, если выполнены три аксиомы "--- ассоциативность, наличие нейтрального элемента, наличие обратного элемента. Группа является \textbf{абелевой}, если операция коммутативна.
\end{itemize}

\end{document}