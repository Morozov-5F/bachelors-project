% !TEX root = ../main.tex
\documentclass[../main.tex]{subfiles}

\begin{document}

В данном приложении представлен листинги процедур, реализующие этапы представленного в пункте 3 алгоритма решения задачи. Для написания программ использовался язык {\ttfamily Wolfram Language}. 

\begin{lstlisting}
(*% Применяет аффинное преобразование к поверхности sruface %*)
TransformSurface[surface_] :=
 Module[{s = surface, st, tr = CreateTransform[]},
  st = s /. {Subscript[x, 1] -> ComplexExpand@Re@Subscript[Z, 1]@tr, 
     Subscript[y, 1] -> ComplexExpand@Im@Subscript[Z, 1]@tr,
     	Subscript[x, 2] -> ComplexExpand@Re@Subscript[Z, 2]@tr, 
     Subscript[y, 2] -> ComplexExpand@Im@Subscript[Z, 2]@tr,
     u -> ComplexExpand@Re@W@tr, v -> ComplexExpand@Im@W@tr};
  Evaluate@st
  ]
\end{lstlisting}

\begin{lstlisting}
(*% Приводит матрицу к ступенчатому виду, принимая в расчет, что переменные variables не являются нулевыми %*)
AdaptiveElmination[matrix_, variables_] :=
  Module[{m = matrix, perms = {}, dim = Dimensions@matrix,
    h = Dimensions@matrix[[1]], w = Dimensions@matrix[[2]],
    vars = Flatten@{variables}, pattern, pos, i},
   pattern = Except[0, _Integer | _?(Variables@# === vars &)];
   For[i = 1, i < Min[w, h], i++,
    pos = Flatten@Position[m[[i ;;, i ;;]], pattern, 2, 1];
    If[pos === {}, Break[], pos += {i - 1, i - 1}];
    m[[{i, pos[[1]]}]] = m[[{pos[[1]], i}]];
    m[[All, {i, pos[[2]]}]] = m[[All, {pos[[2]], i}]];
    m[[i + 1 ;;]] = Factor@MatrixReduceRow[m, i];
    ];
   m
   ];
\end{lstlisting}

\begin{lstlisting}
(*% Печатает матрицу с номерами столбцов и строк %*)
PrintMatrix[matrix_] :=
 Module[{m = matrix, dim = Dimensions@matrix},
  MatrixForm[m, TableHeadings -> {Range@dim[[1]], Range@dim[[2]]}]
  ]
\end{lstlisting}

\begin{lstlisting}
(*% Обнуляет столбец ниже элемента (row, row) в матрице %*)
MatrixReduceRow[matrix_, row_] :=
 Module[{m = matrix, r = row},
  m = # - #[[row]]*matrix[[row]]/matrix[[row, row]] & /@ 
    matrix[[row + 1 ;;]];
  m
  ]
\end{lstlisting}

\begin{lstlisting}
(*% Создает матрицу аффинного преобразования в C3 %*)
CreateTransform[] :=
 Module[{at},
  at = ({{Subscript[A, 1], Subscript[A, 2], Subscript[A, 
        3]}, {Subscript[B, 1], Subscript[B, 2], Subscript[B, 
        3]}, {Subscript[C, 1], Subscript[C, 2], Subscript[C, 3]}}).
     ({{Subscript[z, 1]}, {Subscript[z, 2]}, {w}}) + ({{Subscript[P, 
       1]}, {Subscript[P, 2]}, {q}});
  at = at /. transformSubstitutions;
  Expand@at
  ]
\end{lstlisting}

\end{document}