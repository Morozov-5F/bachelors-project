% !TEX root = ../main.tex
\documentclass[../main.tex]{subfiles}
\newtheorem*{lie_theorem}{Теорема (см~\cite{lie}, с. 82)}
\begin{document}

\subsection{Группы Ли аффинных преобразований в $\mathbb{C}^3$}
Группой Ли называется топологическая группа, если она является параметрической и если функция, задающая закон умножения, является вещественно-аналитичной. Иными словами, зависимость отдельных преобразований от параметров в группе описывается аналитичной функцией.

Действие произвольной подгруппы Ли группы аффинных преобразований пространства $\mathbb{C}^3$ в координатах этого пространства можно задать в матричном виде:
\begin{equation}\label{eq:affine_transform}
\begin{pmatrix}
z_1^* \\
z_2^* \\
w^*
\end{pmatrix} = 
\begin{pmatrix}
A_1(\mathbf t) & A_2(\mathbf t) & A_3(\mathbf t) \\
B_1(\mathbf t) & B_2(\mathbf t) & B_3(\mathbf t) \\
C_1(\mathbf t) & C_2(\mathbf t) & C_3(\mathbf t) \\
\end{pmatrix}
\cdot
\begin{pmatrix}
z_1 \\
z_2 \\
w
\end{pmatrix}
+
\begin{pmatrix}
P_1(\mathbf t) \\
P_2(\mathbf t) \\
q(\mathbf t)
\end{pmatrix},
\end{equation}
где
\begin{equation*}
\begin{pmatrix}
A_1(\mathbf t) & A_2(\mathbf t) & A_3(\mathbf t) \\
B_1(\mathbf t) & B_2(\mathbf t) & B_3(\mathbf t) \\
C_1(\mathbf t) & C_2(\mathbf t) & C_3(\mathbf t) \\
\end{pmatrix}
\end{equation*}
"--- вращательная, а
\begin{equation*}
\begin{pmatrix}
P_1(\mathbf t) \\
P_2(\mathbf t) \\
q(\mathbf t)
\end{pmatrix}
\end{equation*}
"--- сдвиговая компонеты отдельного преобразования из данной группы, $\mathbf t = (t_1, \hdots, t_n)$ "--- вещественные параметры группы, $n$ "--- размерность группы.

Исследуя группы Ли преобразований больших размерностей, естественно представлять многопараметрическую группу как совокупность независимых однопараметрических групп преобразований. В таком случае, обозначим через $F_t$ однопараметрическую группу Ли аффинных преобразований, сохраняющих эту поверхность. Она содержит, в частности, тождественное преобразование, и удобно считать, что оно  к $t = 0$, т.е. $F_0 = \textrm{Id}$. Как следует из~(\ref{eq:affine_transform}), уравнения, определяющие эту группу будут иметь вид:
\begin{equation}\label{eq:affine_transform_eq}
  \begin{cases}
     z_1 &=~A_{1}(t) z_1^* + A_{2}(t) z_2^* + A_{3}(t) w^* + P_1(t) \\
     z_1 &=~B_{1}(t) z_1^* + B_{2}(t) z_2^* + B_{3}(t) w^* + P_2(t) \\
     w   &=~C_{1}(t) z_1^* + C_{2}(t) z_2^* + C_{3}(t) w^* + q(t)
  \end{cases},
\end{equation}
где $A_{i}(t), B_{i}(t), C_{i}(t), P_i(t)$ и $q(t)$ "--- некоторые аналитические комплексные функции от вещественного аргумента. При переходе от 
%% Записать просто линейными формами, типа тривиально показывается
\begin{equation}\label{eq:coordinates_transform}
  \begin{cases}
     x_1 &=~l_1(x_1^*, y_1^*, x_2^*, y_2^*, u^*, v^*)\\
     y_1 &=~l_2(x_1^*, y_1^*, x_2^*, y_2^*, u^*, v^*)\\
     x_2 &=~l_3(x_1^*, y_1^*, x_2^*, y_2^*, u^*, v^*)\\
     y_2 &=~l_4(x_1^*, y_1^*, x_2^*, y_2^*, u^*, v^*)\\
     u   &=~l_5(x_1^*, y_1^*, x_2^*, y_2^*, u^*, v^*)\\
     v   &=~l_6(x_1^*, y_1^*, x_2^*, y_2^*, u^*, v^*)\\
  \end{cases},
\end{equation}
где $l_i(x_1^*, y_1^*, x_2^*, y_2^*, u^*, v^*)$ "--- линейная форма. Явный вид этих форм не приводится, так как его легко получить из выражения~(\ref{eq:affine_transform_eq}).
%где $a_{i, 1} = \Re\{A_i(t)\}$, $a_{i, 2} = \Im\{A_i(t)\}$, $b_{i, 1} = \Re\{B_i(t)\}$, $b_{i, 2} = \Im\{B_i(t)\}$, $c_{i, 1} = \Re\{C_i(t)\}$, $c_{i, 2} = \Im\{C_i(t)\}$, $p_{i, 1} = \Re\{P_i(t)\}$, $p_{i, 2} = \Im\{P_i(t)\}$, $q_1 = \Re\{q(t)\}$ и $q_2 = \Im\{q(t)\}$ "--- вещественные функции от аргумента $t$ (опущен для компактности записи).

\subsection{Инфинитезимальные преобразования}
Важную роль в теории непрерывных групп (Ли) играют инфинитезимальные преобразования. Согласно Софусу Ли: <<Инфинитезимальным преобразованием называется преобразование из однопараметрической группы $F_t$, параметр которого имеет бесконечно малое значение>>. Иными словами, компонентами этого преобразования являются производные уравнений группы $F_t$ в точке $t = 0$:
\begin{equation*}
	\begin{cases}
     \zeta_1 &=A^{\prime}_{1}(0) z_1^* + A^{\prime}_{2}(0) z_2^* + A^{\prime}_{3}(0) w^* + P^{\prime}_1(0) \\
     \zeta_2 &=B^{\prime}_{1}(0) z_1^* + B^{\prime}_{2}(0) z_2^* + B^{\prime}_{3}(0) w^* + P^{\prime}_2(0) \\
     \zeta_3 &=C^{\prime}_{1}(0) z_1^* + C^{\prime}_{2}(0) z_2^* + C^{\prime}_{3}(0) w^* + q^{\prime}(0)
  \end{cases}
\end{equation*}
Удобно записывать коэффициенты этого преобразования в матричном виде:
\begin{equation}\label{eq:infinitesimal_matrix}
\begin{pmatrix}
\alpha_{1,1} + i\cdot\alpha_{1,2} & \alpha_{2,1} + i\cdot\alpha_{2,2} & \alpha_{3,1} + i\cdot\alpha_{3,2} & \sigma_{1,1} + i\cdot\sigma_{1,2} \\
 \beta_{1,1} +  i\cdot\beta_{1,2} &  \beta_{2,1} +  i\cdot\beta_{2,2} &  \beta_{3,1} +  i\cdot\beta_{3,2} & \sigma_{2,1} + i\cdot\sigma_{2,2} \\
\gamma_{1,1} + i\cdot\gamma_{1,2} & \gamma_{2,1} + i\cdot\gamma_{2,2} & \gamma_{3,1} + i\cdot\gamma_{3,2} & \sigma_{3,1} + i\cdot\sigma_{3,2} \\
\end{pmatrix},
\end{equation}
где $\alpha_{i,1} = \Re\{A^{\prime}_{i}(0)\}$, $\alpha_{i,2} = \Im\{A^{\prime}_{i}(0)\}$, $\beta_{i,1} = \Re\{B^{\prime}_{i}(0)\}$, $\beta_{i,2} = \Im\{B^{\prime}_{i}(0)\}$, $\gamma_{i,1} = \Re\{C^{\prime}_{i}(0)\}$, $\gamma_{i,2} = \Im\{C^{\prime}_{i}(0)\}$, $\sigma_{i,1} = \Re\{P^{\prime}_{i}(0)\}$, $\sigma_{i,2} = \Im\{P^{\prime}_{i}(0)\}$, $\sigma_{3,1} = \Re\{q^{\prime}(0)\}$, $\sigma_{3,2} = \Im\{q^{\prime}(0)\}$ "--- вещественные и мнимые компоненты производных функций из~(\ref{eq:affine_transform_eq}) в точке $t = 0$.

Инфинитезимальное преобразование ставит в соответствие каждой точке $p$ некоторой поверхности $M$ из $\mathbb{C}^3$ вектор, касательный к $p$. Иными словами, оно показывает направление движения точки $p$ в результате действия группы $F_t$ для бесконечно малых значений параметра $t$~\cite{lie}.

\textbf{Замечание}: в современной терминологии переход к инфинитезимальным преобразованиям поверхности означает рассмотрение векторных полей, касательных многообразию, соответствующих группе $F_t$.

Далее потребуются теорема из теории групп преобразований Ли. Она будет приведена без доказательства.
\begin{lie_theorem}\label{thm:lie}
Если $r$ инфинитезимальных преобразований являются независимыми друг от друга, а $t_1,\hdots,t_r$ "--- произвольные параметры, то совокупность всех однопараметрических групп, порождаемых этими преобразованиями, образует семейство преобразований, в котором все $r$  параметров $t_1,\hdots,t_r$ являются существенными.
\end{lie_theorem}

Таким образом, число независимых инфинитезимальных преобразований равно размерности порождаемой ими группы Ли. 

Получение однопараметрических групп достигается интегрированием порождающего её инфинитезимального преобразования. В простейшем случае, это означает вычисление матричной экспоненты от произведения матрицы инфинитезимального преобразования~(\ref{eq:infinitesimal_matrix}), дополненной нулевой строкой до квадратной и умноженной и вещественного параметра $t$:
$$
	F_t = \exp(Z \cdot t)
$$
Подробнее о процедуре интегрирования наборов независимых инфинитезимальных преобразований (или, что то же самое, алгебр векторных полей) написано в работах~\cite{loboda_hodarev},~\cite{fels}.

\subsection{Системы полиномиальных уравнений}
В ходе решения поставленной задачи возникают системы полиномиальных уравнений. 

Пусть дана некоторая вещественная гиперповерхность $M$ из семейства~(\ref{eq:initial}) с определяющей функцией $$\Phi(x_1, x_2, y_1, y_2, v) = v x_2 + Q(x_1, y_1, x_2, y_2) - x_2 (\mu x_2^2 + \nu y_2^2).$$ Пусть также $F_t$ "--- однопараметрическая группа аффинных преобразований в $\mathbb{C}^3$, $F_0 = \mathrm{Id}$.
Под действием группы $F_t$ определяющая функция примет следующий вид:
\begin{multline*}
F_t(\Phi) = l_1(x_1^*, y_1^*, x_2^*, y_2^*, u^*, v^*) \cdot l_6(x_1^*, y_1^*, x_2^*, y_2^*, u^*, v^*) + \tilde{Q}(x_1^*, y_1^*, x_2^*, y_2^*, u^*, v^*) - \\
- l_3(x_1^*, y_1^*, x_2^*, y_2^*, u^*, v^*) \cdot \left[\mu \cdot l_4(x_1^*, y_1^*, x_2^*, y_2^*, u^*, v^*){}^2 + \nu \cdot l_4(x_1^*, y_1^*, x_2^*, y_2^*, u^*, v^*){}^2\right],
\end{multline*}
где $\tilde{Q}(x_1^*, y_1^*, x_2^*, y_2^*, u^*, v^*)$ "--- новая квадратичная форма, полученная после действия группы. Далее в работе символ ${}^*$ при координатах будет опущен для компактности.  Введём следующие обозначения:
\begin{equation*}
F_t(\Phi) = T(\mathrm{\RN{1}}) + T(\mathrm{\RN{2}}) - T(\mathrm{\RN{3}}),
\end{equation*}
где
\begin{align*}
T(\mathrm{\RN{1}}) &=
\begin{multlined}[t]
\left(x_1 c_{1,2}+y_1 c_{1,1}+x_2 c_{2,2}+y_2 c_{2,1} + u c_{3,2}+v c_{3,1}+q_2\right) \times \\
\times \left(x_1 b_{1,1}-y_1 b_{1,2}+x_2 b_{2,1}-y_2 b_{2,2} + u b_{3,1}-v b_{3,2}+p_{2,1}\right),
\end{multlined}\\
T(\mathrm{\RN{2}}) &=
\begin{multlined}[t]
\ \tilde{Q}(x_1,y_1,x_2,y_2,u,v),
\end{multlined}\\
T(\mathrm{\RN{3}}) &=
\begin{multlined}[t]
\left(x_1 b_{1,1}-y_1 b_{1,2}+x_2 b_{2,1}-y_2 b_{2,2} + u b_{3,1}-v b_{3,2} + p_{2,1}\right) \times \\
\times \left[\mu  \left(x_1 b_{1,1}-y_1 b_{1,2}+x_2 b_{2,1}-y_2 b_{2,2} + u b_{3,1}-v b_{3,2}+p_{2,1}\right){}^2 + \right. \\
+ \nu \left. \left(x_1 b_{1,2}+y_1 b_{1,1}+x_2 b_{2,2}+y_2 b_{2,1} + u b_{3,2}+v b_{3,1} + p_{2,2}\right){}^2 \right],
\end{multlined}
\end{align*}
$a_{i, 1} = \Re\{A_i(t)\}$, $a_{i, 2} = \Im\{A_i(t)\}$, $b_{i, 1} = \Re\{B_i(t)\}$, $b_{i, 2} = \Im\{B_i(t)\}$, $c_{i, 1} = \Re\{C_i(t)\}$, $c_{i, 2} = \Im\{C_i(t)\}$, $p_{i, 1} = \Re\{P_i(t)\}$, $p_{i, 2} = \Im\{P_i(t)\}$, $q_1 = \Re\{q(t)\}$ и $q_2 = \Im\{q(t)\}$ "--- вещественные функции от аргумента $t$ (опущен для компактности записи).

Из~(\ref{eq:preservation}) следует, что факт сохранения определяющей функции $\Phi$ поверхности $M$ из класса~(\ref{eq:initial}) будет описываться следующим уравнением:
\begin{equation}\label{eq:preservation_m}
F_t\left(\Phi(z, \overline{z}, v)\right) \equiv \psi(z, \overline{z}, w, \overline{w}, t) \cdot \Phi(z, \overline{z}, v).
\end{equation}

Для того, чтобы перейти от аффинных преобразований к инфинитезимальным, продифференцируем выражение~(\ref{eq:preservation_m}) по параметру $t$ в точке $t = 0$:
\begin{equation}\label{eq:diff}
\left.\frac{\mathrm d}{\mathrm{d}t} F_t\left(\Phi(z, \overline{z}, v)\right)\right|_{t = 0} \equiv \Phi(z, \overline{z}, v) \cdot \left.\frac{\mathrm d}{\mathrm{d}t}\psi(z, \overline{z}, w, \overline{w}, t)\right|_{t=0}
\end{equation}

Тогда:
\begin{align*}
\left.\frac{\mathrm d}{\mathrm{d}t}T(\mathrm{\RN{1}})\right|_{t = 0} &=
\begin{multlined}[t]
\left(x_1 \gamma_{1,2}+y_1 \gamma_{1,1}+x_2 \gamma_{2,2}+y_2 \gamma_{2,1} + u \gamma_{3,2}+v \gamma_{3,1}+\sigma_{3,2}\right) x_2 + \\
 + v \left(x_1 \beta_{1,1}-y_1 \beta_{1,2}+x_2 \beta_{2,1}-y_2 \beta_{2,2} + u \beta_{3,1}-v \beta_{3,2}+\sigma_{2,1}\right),
\end{multlined}\nonumber\\
\left.\frac{\mathrm d}{\mathrm{d}t}T(\mathrm{\RN{2}})\right|_{t = 0} &=
\begin{multlined}[t]
\ \left.\frac{\mathrm d}{\mathrm{d}t}\tilde{Q}(x_1,y_1,x_2,y_2,u,v)\right|_{t=0} = \tilde{Q}^{\prime}(x_1,y_1,x_2,y_2,u,v),
\end{multlined}\\
\left.\frac{\mathrm d}{\mathrm{d}t}T(\mathrm{\RN{3}})\right|_{t = 0} &=
\begin{multlined}[t]
\left(\mu  x_2^2+\nu  y_2^2\right) \times \\
\times\left(\sigma _{2,1}+u \beta _{3,1}-v \beta _{3,2}+x_1 \beta _{1,1}+x_2 \beta _{2,1}-y_1 \beta _{1,2}-y_2 \beta_{2,2}\right)+x_2 \times \\
\times\left[2 \mu  x_2 \left(\sigma _{2,1}+u \beta _{3,1}-v \beta _{3,2}+x_1 \beta _{1,1}+x_2 \beta _{2,1}-y_1 \beta _{1,2}-y_2 \beta_{2,2}\right)+\right. \\
+ \left.2 \nu  y_2 \left(\sigma _{2,2}+u \beta _{3,2}+v \beta _{3,1}+x_1 \beta _{1,2}+x_2 \beta _{2,2}+y_1 \beta _{1,1}+y_2 \beta_{2,1}\right)\right],\nonumber
\end{multlined}
\end{align*}

Заметим, что левая часть выражения~(\ref{eq:diff}) является действием инфинитезимального преобразования, порождающего группу $F_t$, на поверхность $M$. Для того, чтобы найти это преобразование, необходимо произвести сужение выражения~(\ref{eq:diff}) на эту поверхность. Для этого выразим переменную $v$ из выражения~(\ref{eq:initial}):
\begin{equation}
v = -\frac{Q(x_1, y_1, x_2, y_2)}{x_2} + \mu x_2^2 + \nu y_2^2.
\end{equation}
Так как сужение определяющей функции поверхности на саму поверхность дает ноль, справедливо следующее тождество:
\begin{equation}\label{eq:main_stuff}
\left. \left\{\left.\frac{\mathrm d}{\mathrm{d}t} F_t\left(\Phi(z, \overline{z}, v)\right)\right|_{t = 0} \right\} \right|_M \equiv 0
\end{equation}

% поправить %
В результате данной подстановки возникает рациональная функция, общий вид которой, вообще говоря, зависит от квадратичной формы $Q$. Проиллюстрируем данный факт на примере компоненты $T(\mathrm{\RN{1}})$:
\begin{multline}
\left.\left\{ \left.\frac{\mathrm d}{\mathrm{d}t}T(\mathrm{\RN{1}})\right|_{t = 0}\right\} \right|_M =
\left(-\frac{Q\left(x_1,y_1,x_2,y_2\right)}{x_2} + \mu x_2^2 + \nu y_2^2\right) \times \\
\times \left( \sigma_{2,1} + u \beta_{3,1} + x_1 \beta_{1,1} + x_2 \beta_{2,1} + x_2 \gamma_{3,1} - y_1 \beta_{1,2} - y_2 \beta_{2,2} \right) - \\
- \beta_{3,2} \left(-\frac{Q\left(x_1,y_1,x_2,y_2\right)}{x_2}+\mu  x_2^2+\nu y_2^2\right){}^2 + \\
+ u x_2 \gamma_{3,2} + x_2^2 \gamma_{2,2} + x_1 x_2 \gamma_{1,2}+x_2 \sigma_{3,2}+x_2 y_1 \gamma_{1,1}+x_2 y_2 \gamma_{2,1}
\end{multline}
%%% количественные характеристики слагаемых
\textbf{Замечание}: другие компоненты из-за своей громоздкости вынесены в приложение 1 (см. формулы~(\ref{eq:appendix_t1})~--~(\ref{eq:appendix_t3})) .

%%% Во всех трех выражениях виден общий знаменатель
%%% Более по-честному, переписать все в линейных формах
Отсюда видно, что общий знаменатель выражения~(\ref{eq:main_stuff}) равен $x_2^2$. Умножая выражение~(\ref{eq:main_stuff}) на него, получаем полиномиальное уравнение шестой степени:
\begin{equation*}
S(x_1, y_1, x_2, y_2, u) = \mathrm{ST(\RN{1})} + \mathrm{ST(\RN{2})} + \mathrm{ST(\RN{3})} \equiv 0,
\end{equation*}
где 
\begin{gather*}
\mathrm{ST(\RN 1)} = x_2^2 \cdot \left.\left\{ \left.\frac{\mathrm d}{\mathrm{d}t}T(\mathrm{\RN{1}})\right|_{t = 0}\right\} \right|_M , \ \mathrm{ST(\RN 2)} = x_2 \cdot \left.\left\{ \left.\frac{\mathrm d}{\mathrm{d}t}T(\mathrm{\RN{2}})\right|_{t = 0}\right\} \right|_M, \\
\mathrm{ST(\RN 3)} = - x_2 \cdot \left.\left\{ \left.\frac{\mathrm d}{\mathrm{d}t}T(\mathrm{\RN{3}})\right|_{t = 0}\right\} \right|_M.
\end{gather*}

Раскрывая скобки и приводя подобные слагаемые, становится видно, что коэффициентами при мономах в полиноме $S$ являются линейные комбинации элементов инфинитезимального преобразования, порождающего группу преобразований $F_t$. Известно, что тождественное равенство нулю полинома означает равенство нулю всех коэффициентов при его одночленах. В таком случае, возникает система однородных уравнений, линейных относительно элементов инфинитезимального преобразования. 

Каждое решение полученной системы уравнений есть инфинитезимальное преобразование. Таким образом, размерность пространства решений данной системы равно количеству независимых инфинитезимальных преобразований. Из теоремы, сформулированной ранее, следует, что размерность подгруппы группы аффинных преобразований, сохраняющих поверхности из исследуемого класса равна числу независимых инфинитезимальных преобразований, а значит и размерности пространства решений полученной однородной системы линейных уравнений.

Как известно, размерность пространства решений однородной системы уравнений равна $n - r$, где $n$ "--- число неизвестных в системе, а $r$ "--- ранг основной матрицы системы~\cite{costrikin_va1}. Таким образом, задача определения размерности группы аффинных преобразований, сохраняющих поверхности из семейства~(\ref{eq:initial}), сводится к определению ранга основной матрицы полученной системы уравнений, а размерность группы будет равна
\begin{equation}
\dim G = 24 - \mathrm{rank}~W,
\end{equation}
где $W$ "--- основная матрица системы. 

Несмотря на то, что рассматриваются системы линейных уравнений, коэффициентами при уравнениях выступают многочлены от коэффициентов квадратичной формы $Q(x_1, y_1, x_2, y_2)$ и параметров $\mu$ и $\nu$. При изучении ранга матрицы, приходится иметь дело с рассмотрением полиномиальных уравнений.

\end{document}