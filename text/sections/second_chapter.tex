% !TEX root = ../main.tex
\documentclass[../main.tex]{subfiles}
\begin{document}

\subsection{Группы Ли аффинных преобразований в $\mathbb{C}^3$}
При обсуждении свойств симметрий, удобно использовать группы Ли. Группой Ли называется топологическая группа, если она является параметрической и если функция, задающая закон умножения, является вещественно-аналитичной. Иными словами, зависимость отдельных преобразований в группе описывается аналитично йфункцией.

Действие группы Ли аффинных преобразований в координатах пространства $\mathbb{C}^3$ можно задать в матричном виде:
\begin{equation}\label{eq:affine_transform}
\begin{pmatrix}
A_1(\mathbf t) & A_2(\mathbf t) & A_3(\mathbf t) \\
B_1(\mathbf t) & B_2(\mathbf t) & B_3(\mathbf t) \\
C_1(\mathbf t) & C_2(\mathbf t) & C_3(\mathbf t) \\
\end{pmatrix}
\cdot
\begin{pmatrix}
z_1 \\
z_2 \\
w
\end{pmatrix}
+
\begin{pmatrix}
P_1(\mathbf t) \\
P_2(\mathbf t) \\
q(\mathbf t)
\end{pmatrix},
\end{equation}
где
\begin{equation*}
\begin{pmatrix}
A_1(\mathbf t) & A_2(\mathbf t) & A_3(\mathbf t) \\
B_1(\mathbf t) & B_2(\mathbf t) & B_3(\mathbf t) \\
C_1(\mathbf t) & C_2(\mathbf t) & C_3(\mathbf t) \\
\end{pmatrix}
\end{equation*}
"--- вращательная, а
\begin{equation*}
\begin{pmatrix}
P_1(\mathbf t) \\
P_2(\mathbf t) \\
q(\mathbf t)
\end{pmatrix}
\end{equation*}
"--- сдвиговая компонеты отдельного преобразования из данной группы, $\mathbf t = (t_1, \hdots, t_n)$ "--- вещественные параметры группы, $n$ "--- размерность группы.

Исследуя группы Ли преобразований больших размерностей, естественно представлять многопараметрическую группу как совокупность независимых однопараметрических групп преобразований. В таком случае, обозначим через $F_t$ однопараметрическую группу Ли аффинных преобразований, сохраняющих эту поверхность. Пусть также эта группа содержит тождественное преобразование при $t = 0$, т.е. $F_0 = \textrm{Id}$. Как следует из~(\ref{eq:affine_transform}), уравнения, определяющие эту группу будут иметь вид:
\begin{equation}
  \begin{cases}
     z_1 &=~A_{1}(t) z_1^* + A_{2}(t) z_2^* + A_{3}(t) w^* + P_1(t) \\
     z_1 &=~B_{1}(t) z_1^* + B_{2}(t) z_2^* + B_{3}(t) w^* + P_2(t) \\
     w   &=~C_{1}(t) z_1^* + C_{2}(t) z_2^* + C_{3}(t) w^* + q(t)
  \end{cases},
\end{equation}
где $A_{i}(t), B_{i}(t), C_{i}(t), P_i(t)$ и $q(t)$ "--- некоторые аналитические комплексные функции от вещественного аргумента. Аналитичность функций возможна, так как группа $F_t$ является группой Ли. Для каждой компоненты координат в комплексном пространстве справедливы следующие уравнения:
\begin{equation}\label{eq:coordinates_transform}
  \begin{cases}
     x_1 &=~x_1 a_{1,1} - y_1 a_{1,2} + x_2 a_{2,1} - y_2 a_{2,2} + u a_{3,1} - v a_{3,2} + p_{1,1}\\
     y_1 &=~x_1 a_{1,2} + y_1 a_{1,1} + x_2 a_{2,2} + y_2 a_{2,1} + u a_{3,2} + v a_{3,1} + p_{1,2}\\
     x_2 &=~x_1 b_{1,1} - y_1 b_{1,2} + x_2 b_{2,1} - y_2 b_{2,2} + u b_{3,1} - v b_{3,2} + p_{2,1}\\
     y_2 &=~x_1 b_{1,2} + y_1 b_{1,1} + x_2 b_{2,2} + y_2 b_{2,1} + u b_{3,2} + v b_{3,1} + p_{2,2}\\
     u   &=~x_1 c_{1,1} - y_1 c_{1,2} + x_2 c_{2,1} - y_2 c_{2,2} + u c_{3,1} - v c_{3,2} + q_1\\
     v   &=~x_1 c_{1,2} + y_1 c_{1,1} + x_2 c_{2,2} + y_2 c_{2,1} + u c_{3,2} + v c_{3,1} + q_2\\
  \end{cases},
\end{equation}
где $a_{i, 1} = \Re\{A_i(t)\}$, $a_{i, 2} = \Im\{A_i(t)\}$, $b_{i, 1} = \Re\{B_i(t)\}$, $b_{i, 2} = \Im\{B_i(t)\}$, $c_{i, 1} = \Re\{C_i(t)\}$, $c_{i, 2} = \Im\{C_i(t)\}$, $p_{i, 1} = \Re\{P_i(t)\}$, $p_{i, 2} = \Im\{P_i(t)\}$, $q_1 = \Re\{q(t)\}$ и $q_2 = \Im\{q(t)\}$ "--- вещественные функции от аргумента $t$ (опущен для компактности записи).

\subsection{Инфинитезимальные преобразования}
Важную роль в теории непрерывных групп (Ли) играют инфинитезимальные преобразования. Инфинитезимальным преобразованием называется преобразование из однопараметрической группы $F_t$, параметр которого имеет бесконечно малое значение. Иными словами, компонентами этого преобразования являются производные уравнений группы $F_t$ в точке $t = 0$:
\begin{equation*}
	\begin{cases}
     z_1 &=~z^*_1 + (A^{\prime}_{1}(0) z_1^* + A^{\prime}_{2}(0) z_2^* + A^{\prime}_{3}(0) w^* + P^{\prime}_1(0))\delta t \\
     z_2 &=~z^*_2 + (B^{\prime}_{1}(0) z_1^* + B^{\prime}_{2}(0) z_2^* + B^{\prime}_{3}(0) w^* + P^{\prime}_2(0))\delta t \\
     w   &=~w^*   + (C^{\prime}_{1}(0) z_1^* + C^{\prime}_{2}(0) z_2^* + C^{\prime}_{3}(0) w^* + q^{\prime}(0))\delta t
  \end{cases}
\end{equation*}
Удобно записывать коэффициенты этого преобразования в матричном виде:
\begin{equation}
\begin{pmatrix}
\alpha_{1,1} + i\cdot\alpha_{1,2} & \alpha_{2,1} + i\cdot\alpha_{2,2} & \alpha_{3,1} + i\cdot\alpha_{3,2} & \sigma_{1,1} + i\cdot\sigma_{1,2} \\
 \beta_{1,1} +  i\cdot\beta_{1,2} &  \beta_{2,1} +  i\cdot\beta_{2,2} &  \beta_{3,1} +  i\cdot\beta_{3,2} & \sigma_{2,1} + i\cdot\sigma_{2,2} \\
\gamma_{1,1} + i\cdot\gamma_{1,2} & \gamma_{2,1} + i\cdot\gamma_{2,2} & \gamma_{3,1} + i\cdot\gamma_{3,2} & \sigma_{3,1} + i\cdot\sigma_{3,2} \\
\end{pmatrix},
\end{equation}
где $a^{\prime}_{i,j}(0) = \alpha_{i,j}$, $b^{\prime}_{i,j}(0) = \beta_{i,j}$, $c^{\prime}_{i,j}(0) = \gamma_{i,j}$, $p^{\prime}_{i,j}(0) = \sigma_{i,j}$, $q^{\prime}_{j}(0) = \sigma_{3,j}$ "--- производные функций из~(\ref{eq:coordinates_transform}) в точке $t = 0$.

Инфинитезимальное преобразование ставит в соответствие каждой точке $p$ некторой поверхности $M$ из $\mathbb{C}^3$ вектор, касательный к $p$. Иными словами, оно показывает направление движения точки $p$ в результате действия группы $F_t$ для бесконечно малых значений параметра $t$~\cite{lie}.

\textbf{Замечание}: TODO векторные поля и связь с ними

Далее потребуются две теоремы из теории групп преобразований Ли. Они будут приведены без доказательств.
\begin{theorem}
Если $r$ инфинитезимальных преобразований являются независимыми друг от дргуа, а $t_1,\hdots,t_r$ "--- произвольные параметры, то совокупность всех однопараметрических групп, порождаемых этими преобразованиями, образует семейство преобразований, в котором все $r$  параметров $t_1,\hdots,t_r$ являются существенными.
\end{theorem}

\subsection{Системы полиномиальных уравнений}
TODO

\end{document}