% !TEX root = ../main.tex
\documentclass[../main.tex]{subfiles}
\begin{document}

В данном разделе приведены описания процедур и функций, которые позволяют реализовать алгоритм нахождения размерности групп аффинных преобразований, предложенный в пункте 3 данной работы. 

\subsection{Основные процедуры}

\begin{itemize}

\item {\ttfamily TransformSurface} "--- функция, применяющее аффинное преобразование в трехмерном комплексном пространстве к данному уравнению поверхности. 

\item {\ttfamily AdaptiveElimination} "--- функция, приводящая заданную матрицу к ступенчатому виду. Особенность этой процедуры в том, что, помимо матриц, на вход подается список переменных, которые считаются отличными от нуля и элементы, которые содержат в себе только эти переменные, могут стоять на главной диагонали матрицы. Эта процедура была необходима, потому что пакет {\ttfamily Wolfram Mathematica} не позволяет приводить матрицу к ступенчатому виду, принимая в расчет какие-либо предположения относительно переменных, кроме наиболее общих (комлексное, вещественное число и т.п.).

\item {\ttfamily GetSystem} "---  aaa

\item {\ttfamily IntegrateIT} "---  aaa

\end{itemize}

\subsection{Вспомогательные процедуры}
\begin{itemize}
\item {\ttfamily CreateTransform} "---

\item {\ttfamily PrintMatrix} "---

\item {\ttfamily MatrixReduceRow} "---

\end{itemize}

\end{document}