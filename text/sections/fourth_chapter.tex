% !TEX root = ../main.tex
\documentclass[../main.tex]{subfiles}
\begin{document}

В данном разделе приведены описания процедур и функций, которые позволяют реализовать алгоритм нахождения размерности групп аффинных преобразований, который был предложен в разделе 3 данной работы. 

В связи с трудоемкостью процедуры и большим объемом вычислений, сложно обойтись без использования пакетов символьных вычислений. С помощью пакета символьной математики {\ttfamily Wolfram Mathematica} были разработаны функции, способные выполнять отдельные этапы алгоритма решения поставленной задачи.

\textbf{Замечание}: пакет {\ttfamily Wolfram Mathematica} имеет очень богатую {стандартную} библиотеку функций, что довольно сильно облегчает работу. Для разработки программы использовались стандартные средства сопоставления по образцу, подстановки в выражение ({\ttfamily Replace}), решения уравнений ({\ttfamily {Sol-}{ve}}) и многие другие.

Листинги представленных функций и примеры их использования можно найти в приложении Б.

\subsection{Основные функции}

\begin{itemize}

\item {\ttfamily TransformSurface} "--- функция, применяющая аффинное преобразование в трехмерном комплексном пространстве к данному уравнению поверхности. Входными параметрами данной функции является уравнение поверхности и матрица аффинного преобразования, которое должно быть применено к этой поверхности.

\item {\ttfamily AdaptiveElimination} "--- функция, приводящая заданную матрицу к ступенчатому виду. Особенность этой процедуры в том, что, помимо матриц, на вход подается список переменных, которые считаются отличными от нуля и элементы, которые содержат в себе только эти переменные, могут стоять на главной диагонали матрицы. Эта процедура была необходима, потому что пакет {\ttfamily Wolfram Mathematica} не позволяет приводить матрицу к ступенчатому виду, принимая в расчет какие-либо предположения относительно переменных, кроме наиболее общих (комлексное, вещественное число и т.п.).

\item {\ttfamily GetSystem} "---  функция для получения системы линейных уравнений на элементы инфинитезимального преобразования для данной поверхности. На вход подается уравнение поверхности. Функция возвращает список из уравнений полученной линейной системы.

\item {\ttfamily GetInfinitesimalTransforms} "--- функция для решения системы линейных уравнений, полученной в результате применения функции {\ttfamily GetSystem}. На вход принимается список уравнений. Функция возвращает список из матриц независимых инфинитезимальных преобразований.

\item {\ttfamily IntegrateTransforms}~ "--- ~ функция~~ для ~~интегрирования ~~набора\\ из независимых инфинитезимальных преобразований. Возвращает список однопараметрических групп, полученных в результате интегрирования. 

\item {\ttfamily CheckPreservation} "--- функция, проверяющая факт сохранения поверхности однопараметрической группой преобразований. Входные параметры "--- уравнение поверхности и однопараметрическая группа в матричном виде. Возвращает {\ttfamily True}, если поверхность сохраняется, {\ttfamily False} в ином случае.

\end{itemize}

\subsection{Вспомогательные процедуры}
\begin{itemize}
\item {\ttfamily CreateTransform} "--- Генерирует однопарметрическую группу преобразований в матричном виде~(\ref{eq:affine_transform}). Функция не имеет входных параметров и возвращает матрицу для произвольной однопараметрической группы аффинных преобразований.

\item {\ttfamily PrintMatrix} "--- Распечатывает матрицу с нумерацией строк и столбцов.

\item {\ttfamily MatrixReduceRow} "--- Функция, осуществляющая шаг приведения матрицы к ступенчатому виду. Входными параметрами является матрица и номер строки.

\item {\ttfamily SwapRows} "--- функция для обмена местами строк в матрице. На вход подается матрица и два списка одинакового размера: в первом находятся исходные позиции строк, во втором "--- конечные. Функция возвращает матрицу с переставленными строками.

\item {\ttfamily SwapColumns} "--- Меняет местами столбцов в матрице. Функция является аналогичной {\ttfamily SwapRows}.

\end{itemize}

\subsection{Результаты тестирования}
В ходе тестирования представленной программы было обнаружено, что в случае произвольных поверхностей с большим количеством неопределенных коэффициентов, не удается получить точную размерность группы, не прибегая к <<ручному режиму>>. Часть инфинитезимальных преобразований теряется в ходе решения системы уравнений с большим количеством параметров.

Поскольку мы не можем знать, какие решения системы будут потеряны, приходится все компьютерные действия сопровождать ручными вычислениями. ~~~При ~~~этом ~~~отдельные ~~~функции, ~~написанные ~~в ~~{\ttfamily Wolfram\\ Mathematica}, выполняют большую часть работы, существенно ускоряя решение задачи и освобождая от естественных ошибок <<ручного режима>>.

\end{document}