% !TEX root = ../main.tex
\documentclass[../main.tex]{subfiles}
\begin{document}
\setcounter{equation}{0}
\renewcommand{\theequation}{П1.\arabic{equation}}
В данном приложении перечислены выражения, которые слишком велики для того, чтобы помещать их напрямую в текст.

\subsection{Компоненты $T(\mathrm{\RN{1}})$, $T(\mathrm{\RN{2}})$ и $T(\mathrm{\RN{3}})$ в сужении на $M$}
\begin{multline}\label{eq:appendix_t1}
\left.\left\{ \left.\frac{\mathrm d}{\mathrm{d}t}T(\mathrm{\RN{1}})\right|_{t = 0}\right\} \right|_M =
\left(-\frac{Q\left(x_1,y_1,x_2,y_2\right)}{x_2} + \mu x_2^2 + \nu y_2^2\right) \times \\
\times \left( \sigma_{2,1} + u \beta_{3,1} + x_1 \beta_{1,1} + x_2 \beta_{2,1} + x_2 \gamma_{3,1} - y_1 \beta_{1,2} - y_2 \beta_{2,2} \right) - \\ 
- \beta_{3,2} \left(-\frac{Q\left(x_1,y_1,x_2,y_2\right)}{x_2}+\mu  x_2^2+\nu y_2^2\right)^2 + \\
+ u x_2 \gamma_{3,2} + x_2^2 \gamma_{2,2} + x_1 x_2 \gamma_{1,2}+x_2 \sigma_{3,2}+x_2 y_1 \gamma_{1,1}+x_2 y_2 \gamma_{2,1}
\end{multline}

\begin{multline}\label{eq:appendix_t2}
\left.\left\{ \left.\frac{\mathrm d}{\mathrm{d}t}T(\mathrm{\RN{2}})\right|_{t = 0}\right\} \right|_M = \textrm{TODO}
\end{multline}

\begin{multline}\label{eq:appendix_t3}
\left.\left\{ \left.\frac{\mathrm d}{\mathrm{d}t}T(\mathrm{\RN{3}})\right|_{t = 0}\right\} \right|_M = \textrm{TODO}
\end{multline}

\end{document}