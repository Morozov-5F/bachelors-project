% !TEX root = ../main.tex
\documentclass[../main.tex]{subfiles}
\begin{document}

В данном разделе будет описана процедура определения размерностей групп аффинных преобразований для семейства поверхностей~(\ref{eq:initial}).
\subsection{Общая схема}
Пусть дана некоторая вещественная гиперповерхность $M$ из семейства~(\ref{eq:initial}) с определяющей функцией $$\Phi(x_1, x_2, y_1, y_2, v) = v x_2 - Q(x_1, y_1, x_2, y_2) - x_2 (\mu x_2^2 + \nu y_2^2).$$ Пусть также $F_t$ "--- однопараметрическая группа аффинных преобразований в $\mathbb{C}^3$, $F_0 = \mathrm{Id}$.
Под действием группы $F_t$ определяющая функция примет следующий вид:
\begin{multline*}
F_t(\Phi) = \left(x_1 c_{1,2}+y_1 c_{1,1}+x_2 c_{2,2}+y_2 c_{2,1} + u c_{3,2}+v c_{3,1}+q_2\right) \times \\
\times \left(x_1 b_{1,1}-y_1 b_{1,2}+x_2 b_{2,1}-y_2 b_{2,2} + u b_{3,1}-v b_{3,2}+p_{2,1}\right) + \tilde{Q}(x_1,y_1,x_2,y_2,u,v) - \\
-\left(x_1 b_{1,1}-y_1 b_{1,2}+x_2 b_{2,1}-y_2 b_{2,2}+ u b_{3,1}-v b_{3,2}+p_{2,1}\right) \times \\
\times \left[\mu  \left(x_1 b_{1,1}-y_1 b_{1,2}+x_2 b_{2,1}-y_2 b_{2,2} + u b_{3,1}-v b_{3,2}+p_{2,1}\right){}^2 + \right. \\
+ \ \left. \nu \left(x_1 b_{1,2}+y_1 b_{1,1}+x_2 b_{2,2}+y_2 b_{2,1} + u b_{3,2}+v b_{3,1} + p_{2,2}\right){}^2 \right],
\end{multline*}
где $\tilde{Q}(x_1, y_1, x_2, y_2)$ "--- новая квадратичная форма, полученная после действия группы. Введём следующие обозначения:
\[
F_t(\Phi) = T(\mathrm{\RN{1}}) + T(\mathrm{\RN{2}}) - T(\mathrm{\RN{3}}),
\]
где
\begin{align*}
T(\mathrm{\RN{1}}) &=
\begin{multlined}[t]
\left(x_1 c_{1,2}+y_1 c_{1,1}+x_2 c_{2,2}+y_2 c_{2,1} + u c_{3,2}+v c_{3,1}+q_2\right) \times \\
\times \left(x_1 b_{1,1}-y_1 b_{1,2}+x_2 b_{2,1}-y_2 b_{2,2} + u b_{3,1}-v b_{3,2}+p_{2,1}\right),
\end{multlined}\\
T(\mathrm{\RN{2}}) &=
\begin{multlined}[t]
\ \tilde{Q}(x_1,y_1,x_2,y_2,u,v),
\end{multlined}\\
T(\mathrm{\RN{3}}) &=
\begin{multlined}[t]
\left(x_1 b_{1,1}-y_1 b_{1,2}+x_2 b_{2,1}-y_2 b_{2,2} + u b_{3,1}-v b_{3,2} + p_{2,1}\right) \times \\
\times \left[\mu  \left(x_1 b_{1,1}-y_1 b_{1,2}+x_2 b_{2,1}-y_2 b_{2,2} + u b_{3,1}-v b_{3,2}+p_{2,1}\right){}^2 + \right. \\
+ \nu \left. \left(x_1 b_{1,2}+y_1 b_{1,1}+x_2 b_{2,2}+y_2 b_{2,1} + u b_{3,2}+v b_{3,1} + p_{2,2}\right){}^2 \right].
\end{multlined}
\end{align*}

Из~(\ref{eq:preservation}) следует, что факт сохранения определяющей функции $\Phi$ поверхности $M$ из класса~(\ref{eq:initial}) будет описываться следующим уравнением:
\begin{equation}\label{eq:preservation_m}
F_t\left(\Phi(z, \overline{z}, v)\right) \equiv \psi(z, \overline{z}, w, \overline{w}, t) \cdot \Phi(z, \overline{z}, v).
\end{equation}

Для того, чтобы перейти от аффинных преобразований к инфинитезимальным, продифференцируем выражение~(\ref{eq:preservation_m}) по параметру $t$ в точке $t = 0$:
\begin{equation}\label{eq:diff}
\left.\frac{\mathrm d}{\mathrm{d}t} F_t\left(\Phi(z, \overline{z}, v)\right)\right|_{t = 0} \equiv \Phi(z, \overline{z}, v) \cdot \left.\frac{\mathrm d}{\mathrm{d}t}\psi(z, \overline{z}, w, \overline{w}, t)\right|_{t=0}
\end{equation}

Тогда:
\begin{align*}
\left.\frac{\mathrm d}{\mathrm{d}t}T(\mathrm{\RN{1}})\right|_{t = 0} &=
\begin{multlined}[t]
\left(x_1 \gamma_{1,2}+y_1 \gamma_{1,1}+x_2 \gamma_{2,2}+y_2 \gamma_{2,1} + u \gamma_{3,2}+v \gamma_{3,1}+\sigma_{3,2}\right) x_2 + \\
 + v \left(x_1 \beta_{1,1}-y_1 \beta_{1,2}+x_2 \beta_{2,1}-y_2 \beta_{2,2} + u \beta_{3,1}-v \beta_{3,2}+\sigma_{2,1}\right),
\end{multlined}\nonumber\\
\left.\frac{\mathrm d}{\mathrm{d}t}T(\mathrm{\RN{2}})\right|_{t = 0} &=
\begin{multlined}[t]
\ \left.\frac{\mathrm d}{\mathrm{d}t}\tilde{Q}(x_1,y_1,x_2,y_2,u,v)\right|_{t=0} = \tilde{Q}^{\prime}(x_1,y_1,x_2,y_2,u,v),
\end{multlined}\\
\left.\frac{\mathrm d}{\mathrm{d}t}T(\mathrm{\RN{3}})\right|_{t = 0} &=
\begin{multlined}[t]
\left(\mu  x_2^2+\nu  y_2^2\right) \times \\
\times\left(\sigma _{2,1}+u \beta _{3,1}-v \beta _{3,2}+x_1 \beta _{1,1}+x_2 \beta _{2,1}-y_1 \beta _{1,2}-y_2 \beta_{2,2}\right)+x_2 \times \\
\times\left[2 \mu  x_2 \left(\sigma _{2,1}+u \beta _{3,1}-v \beta _{3,2}+x_1 \beta _{1,1}+x_2 \beta _{2,1}-y_1 \beta _{1,2}-y_2 \beta_{2,2}\right)+\right. \\
+ \left.2 \nu  y_2 \left(\sigma _{2,2}+u \beta _{3,2}+v \beta _{3,1}+x_1 \beta _{1,2}+x_2 \beta _{2,2}+y_1 \beta _{1,1}+y_2 \beta_{2,1}\right)\right],\nonumber
\end{multlined}
\end{align*}

Заметим, что левая часть выражения~(\ref{eq:diff}) является действием инфинитезимального преобразования, порождающего группу $F_t$, на поверхность $M$. Для того, чтобы найти это преобразование, необходимо произвести сужение выражения~(\ref{eq:diff}) на эту поверхность. Для этого выразим переменную $v$ из выражения~(\ref{eq:initial}):
\begin{equation*}
v = -\frac{Q(x_1, y_1, x_2, y_2)}{x_2} + \mu x_2^2 + \nu y_2^2.
\end{equation*}
Так как сужение определяющей функции поверхности на саму поверхность дает ноль, справедливо следующее тождество:
\begin{equation}\label{eq:main_stuff}
\left. \left\{\left.\frac{\mathrm d}{\mathrm{d}t} F_t\left(\Phi(z, \overline{z}, v)\right)\right|_{t = 0} \right\} \right|_M \equiv 0
\end{equation}

В результате данной подстановки возникает рациональная функция, общий вид которой, вообще говоря, зависит от квадратичной формы $Q$. Проиллюстрируем данный факт на примере компоненты $T(\mathrm{\RN{1}})$:
\begin{multline}
\left.\left\{ \left.\frac{\mathrm d}{\mathrm{d}t}T(\mathrm{\RN{1}})\right|_{t = 0}\right\} \right|_M =
\left(-\frac{Q\left(x_1,y_1,x_2,y_2\right)}{x_2} + \mu x_2^2 + \nu y_2^2\right) \times \\
\times \left( \sigma_{2,1} + u \beta_{3,1} + x_1 \beta_{1,1} + x_2 \beta_{2,1} + x_2 \gamma_{3,1} - y_1 \beta_{1,2} - y_2 \beta_{2,2} \right) - \\
- \beta_{3,2} \left(-\frac{Q\left(x_1,y_1,x_2,y_2\right)}{x_2}+\mu  x_2^2+\nu y_2^2\right){}^2 + \\
+ u x_2 \gamma_{3,2} + x_2^2 \gamma_{2,2} + x_1 x_2 \gamma_{1,2}+x_2 \sigma_{3,2}+x_2 y_1 \gamma_{1,1}+x_2 y_2 \gamma_{2,1}
\end{multline}

\textbf{Замечание}: другие компоненты из-за своей громоздкости вынесены в приложение 1 (см. формулы~(\ref{eq:appendix_t1})~--~(\ref{eq:appendix_t3})) .

Отсюда видно, что общий знаменатель выражения~(\ref{eq:main_stuff}) равен $x_2^2$. Умножая выражение~(\ref{eq:main_stuff}) на него, получаем полиномиальное уравнение:
\begin{equation*}
S(x_1, y_1, x_2, y_2, u) = \mathrm{сокращение} \equiv 0
\end{equation*}



%%%%%%%
\subsection{Произвольная поверхность}
В общем случае, квадратичную форму $Q(x_1, y_1, x_2, y_2)$ можно записать в явном виде:
\begin{align*}
Q(x_1, y_1, x_2, y_2) &= k_1 x_1^2 + k_2 x_2 x_1 + k_3 x_2^2 + k_4 x_1 y_1 + k_5 x_2 y_1 \\
&+ k_6 y_1^2 + k_7 x_1 y_2 + k_8 x_2 y_2 + k_9 y_1 y_2 + k_{10} y_2^2
\end{align*}

Применяя метод, указанный в предыдущем пункте, получим линейную однородную систему из 83 уравнений относительно 24 неизвестных, причем переменная $\sigma_{3,1}$ является <<фиктивной>> "--- в связи с тем, что   Наиболее простые уравнения в системе имеют вид
\begin{equation}
\begin{array}{l}
 k_1 \beta _{3,2}=0 \\
 k_1 \beta _{3,1}=0 \\
 k_1 \sigma _{2,1}=0 \\
 k_4 \beta _{3,1}=0 \\
 k_4 \sigma _{2,1}=0 \\
 k_7 \beta _{3,1}=0 \\
 k_7 \sigma _{2,1}=0 \\
 k_6 \beta _{3,1}=0 \\
 k_6 \sigma _{2,1}=0 \\
 k_9 \beta _{3,1}=0 \\
 k_9 \sigma _{2,1}=0 \\
 k_6 \beta _{3,2}=0 \\
 k_{10} \beta _{3,2}=0 \\
\end{array}
\end{equation}
Отсюда можно выделить два случая:
\begin{enumerate}
	\item Все $k_j$ равны нулю: $k_1 = k_4 = k_6 = k_7 = k_9 = k_{10} = 0$.
	\item Хотя бы один $k_j$ из указанного набора не равен нулю.
\end{enumerate}
В первом случае поверхность приобретает следующий вид:
\begin{equation*}
v x_2 = k_2 x_2 x_1 + k_3 x_2^2 + k_5 x_2 y_1 + k_8 x_2 y_2 + x_2 (\mu x_2^2 + \nu y_2^2).
\end{equation*}
Легко видеть, что данное уравнение можно поделить на $x_2$, что в итоге даст новую поверхность, которая является квадрикой. Рассмотрение поверхностей данного типа не представляет интереса в данной работе, поэтому можно сразу перейти к рассмотрению второго случая.

Предположим, что $k_1 \ne 0 $. В таком случае, приведем элементарными преобразованиями матрицу системы $W$ к ступенчатому виду.

Исходя их этих соображений можно сформулировать следующую теорему:
\begin{theorem} Размерность группы Ли аффинных преобразований, сохраняющих любую поверхность из семейства~(\ref{eq:initial}), удовлетворяет неравенству $1 \le \dim G \le 5$, и каждая из размерностей достижима.
\end{theorem}

%%%%%%%
\subsection{Частный случай семейства}
В связи со сложностью рассмотрения общего случая, был рассмотрен частный случай семейства поверхностей~(\ref{eq:initial}), в котором квадратичная форма $Q$ не зависит от переменных $x_2$ и $y_2$. В таком случае, семейство может быть описано следующим образом:
\begin{equation}\label{eq:special}
v x_2 = k_1 x_1^2 + k_2 x_1 y_1 + k_3 x_2^2 + x_2 (\mu x_2^2 + \nu y_2^2),
\end{equation}

Следует заметить, что за счёт применения невырожденных аффинных преобразований, можно фактически сократить число параметров $k_1, k_2$ и $k_3$ до одного.

В результате применения описанного метода, получается однородная линейная система из 57 уравнений относительно 24 неизвестных.

\begin{theorem} размерность группы Ли аффинных преобразований, сохраняющих любую поверхность вида~(\ref{eq:special}), удовлетворяет неравенствам
$3 \le \dim G \le 5$, причем
\begin{itemize}
	\item $\dim G$ = 3 достигается на поверхностях вида
	\begin{equation}\label{eq:special_3}
		v x_2 = k x_1^2 + y_1^2 + x_2 (\mu x_2^2 + \nu y_2^2),\ k \ne 0, \ k \ne 1;
	\end{equation}
	\item $\dim G$ = 4 достигается на поверхностях вида
	\begin{equation}\label{eq:special_4}
		v x_2 = |z_1|^2 + x_2 (\mu x_2^2 + \nu y_2^2);
	\end{equation}
		\item $\dim G$ = 5 достигается на аффинно-однородных поверхностях следующего вида
	\begin{equation}\label{eq:special_5}
		v x_2 = x_1^2 + x_2 (\mu x_2^2 + \nu y_2^2).
	\end{equation}
\end{itemize}
\end{theorem}

\end{document}