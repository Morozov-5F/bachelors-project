% !TEX root = ../main.tex
\documentclass[../main.tex]{subfiles}
\begin{document}

В данном разделе будет описана процедура определения размерностей групп аффинных преобразований для семейства поверхностей~(\ref{eq:initial}).
\subsection{Общая схема}
Основываясь на сведениях из раздела 2, можно сформулировать следующий алгоритм для нахождения размерности групп аффинных преобразований, сохраняющих поверхности из семейства~(\ref{eq:initial}):
\begin{enumerate}
\item Применить аффинное преобразование~(\ref{eq:affine_transform}) к уравнению поверхности
\item Продифференцировать полученное выражение по параметру группы $t$ в точке $t = 0$.
\item Разрешить уравнение поверхности относительно переменной $v$ и подставить в выражение и предыдущего шага и умножить на возникающий общий знаменатель ($x_2^2$).
\item Составить однородную систему линейных относительно элементов инфинитезимального преобразования уравнений и выписать матрицу этой системы.
\item Размерность группы Ли аффинных преобразований $G$, сохраняющих поверхность из исследуемого семейства, равна $\dim G = 24 - \mathrm{rank}~W$, где $W$ "--- полученная матрица системы.
\end{enumerate}

%%%%%%
\subsection{Частный случай семейства}
В связи со сложностью рассмотрения общего случая, был рассмотрен частный случай семейства поверхностей~(\ref{eq:initial}), в котором квадратичная форма $Q$ не зависит от переменных $x_2$ и $y_2$. В таком случае, семейство может быть описано следующим образом:
\begin{equation}\label{eq:special}
v x_2 = k_1 x_1^2 + k_2 x_1 y_1 + k_3 x_2^2 + x_2 (\mu x_2^2 + \nu y_2^2),
\end{equation}
где $k_1$, $k_2$, $k_3$ "--- вещественные параметры.

Следует заметить, что за счёт применения невырожденных аффинных преобразований, можно фактически сократить число параметров $k_1, k_2$ и $k_3$ до одного:

Применим описанный метод для этого класса поверхностей.

\begin{enumerate}
\item Применим аффинное преобразование к определяющей функции поверхности из семейства~(\ref{eq:special}):
\begin{multline}\label{eq:special_transformed}
F_t(\Phi) = l_6(x_1, y_1, x_2, y_2, u, v) \cdot l_3(x_1, y_1, x_2, y_2, u, v) + k_1 l_1(x_1, y_1, x_2, y_2, u, v){}^2 + \\ + k_2 l_1(x_1, y_1, x_2, y_2, u, v)\cdot l_2(x_1, y_1, x_2, y_2, u, v) + k_3 l_2(x_1, y_1, x_2, y_2, u, v){}^2 - \\ 
- l_3(x_1, y_1, x_2, y_2, u, v) \left[\mu l_3(x_1, y_1, x_2, y_2, u, v){}^2+\nu  l_4(x_1, y_1, x_2, y_2, u, v){}^2\right]
\end{multline}

\item Производная~(\ref{eq:special_transformed}) по $t$ в точке $t = 0$:
\begin{multline}\label{eq:special_diff}
\left.\frac{\mathrm d}{\mathrm{d}t} F_t\left(\Phi\right)\right|_{t = 0} = v l_3^{\prime}(x_1, y_1, x_2, y_2, u, v) + x_2 l_6^{\prime}(x_1, y_1, x_2, y_2, u, v) + \\
+ 2 k_1 x_1 l_1^{\prime}(x_1, y_1, x_2, y_2, u, v) + k_2 y_1 l_2^{\prime}(x_1, y_1, x_2, y_2, u, v) + \\
+ k_2 x_1 l_2^{\prime}(x_1, y_1, x_2, y_2, u, v) +2 k_3 y_1 l_2^{\prime}(x_1, y_1, x_2, y_2, u, v) - \left(\mu x_2^2+\nu  y_2^2\right) \times \\
\times l_3^{\prime}(x_1, y_1, x_2, y_2, u, v) - x_2 \left[2 \mu  x_2 l_3^{\prime}(x_1, y_1, x_2, y_2, u, v) + \right. \\
\left. +2 \nu  y_2 l_4^{\prime}(x_1, y_1, x_2, y_2, u, v)\right]
\end{multline}

\item Разрешим уравнение~(\ref{eq:special}) относительно $v$:
$$
v = \frac{k_1 x_1^2 + k_2 x_1 y_1 + k_3 x_2^2}{x_2} +  \mu x_2^2 + \nu y_2^2
$$
и подставим его в~(\ref{eq:special_diff}), умножая на $x_2^2$. В данном многочлене имеется 57 мономов и 101 слагаемое. Приведем подобные и выпишем по одному примеру одночленов для каждой из степеней:
\begin{align*}
\text{степень 6}&:\ x_1 x_2^2 y_2 \left(k_2 \alpha _{2,1}-2 k_1 \alpha _{2,2}\right) \\
\text{степень 5}&:\ x_1 x_2^4 \left(-2 \mu  \beta _{1,1}+k_2 \mu  \alpha _{3,1}-2 k_1 \mu  \alpha _{3,2}\right) \\ 
\text{степень 4}&:\ x_1 x_2 y_1^2 \left(k_2^2 \alpha _{3,2}-3 k_3 k_2 \alpha _{3,1}+2 k_1 k_3 \alpha _{3,2}+k_2 \beta _{1,2}-k_3 \beta _{1,1}\right) \\
\text{степень 3}&:\ x_2^3 \sigma _{3,2}
\end{align*}
Видно, что есть как довольно простые (степень 3), так и довольно сложные (степень 4) коэффициенты при одночленах.

\item Составим однородную систему линейных уравнений на основе многочлена, полученного в предыдущем пункт. Получается система из 57 уравнений относительно 24 неизвестных. Схему основной матрицы этой системы можно посмотреть в приложении В. 

В качестве примеров, приведем некоторые уравнения из этой системы:
\begin{equation}
\begin{aligned}
k_2^2 \left(-\alpha _{3,1}\right)+3 k_1 k_2 \alpha _{3,2}-2 k_1 k_3 \alpha _{3,1}-k_2 \beta _{1,1}+k_1 \beta _{1,2} &= 0 \\
2 k_2 \alpha _{1,1}-2 k_1 \alpha _{1,2}+2 k_3 \alpha _{1,2}-k_2 \beta _{2,1}-k_2 \gamma _{3,1} &=0 \\
-2 \mu  \beta _{1,1}+k_2 \mu  \alpha _{3,1}-2 k_1 \mu  \alpha _{3,2} &= 0 \\
\left(k_2^2 + 2 k_1 k_3\right) \beta _{3,2} &=0 \\ 
k_2 \alpha _{2,1}-2 k_1 \alpha _{2,2} &= 0 \\
\gamma_{3,2} &= 0
\end{aligned}
\end{equation}
\item Для изучения ранга этой матрицы, приведем её элементарными преобразованиями к ступенчатому виду.
\end{enumerate}

Предположим, что $k_1 \ne 0$. В таком случае, мы можем сделать замену переменных $z^*_1 = z_1
/ \sqrt k_1$, и замену параметров $k^*_2 = k_2 / k_1$, $k^*_3 = k_3 / k_1$, чтобы избавиться от параметра $k_1$ в уравнении поверхности, а значит, и в системе уравнений. Теперь мы можем привести матрицу системы $W$ к следующему квазитреугольному виду:
\begin{equation*}
W \sim
\begin{pmatrix}
A & N \\
0 & B \\
\hdots & \hdots  \\
0 & 0
\end{pmatrix},
\end{equation*}
где матрица $A$ "--- диагональная матрица размерами $14\times14$, $N\in \mathbb{R}^{14\times10}$, $B \in \mathbb{R}^{15\times10}$ и имеет вид (символы ${}^*$ при параметрах $k_2$ и $k_3$ опущены):

\begin{equation}\label{eq:aux_matrix}
{\small\left(
\begin{array}{cccccccccc}
 0 & -\nu  \left(k_2^2-4 k_3\right) & 0 & 0 & 0 & 0 & \nu  k_2 & 0 & 0 & 0 \\
 0 & 0 & 0 & k_2^2-2 k_3+2 & 0 & 0 & 0 & 0 & 0 & 0 \\
 0 & 0 & 0 & 0 & 0 & 0 & -\nu  & 0 & 0 & 0 \\
 0 & 0 & 0 & 0 & \frac{1}{2} \left(k_2^2+4\right) & 0 & 0 & 0 & 0 & -\frac{k_2}{2} \\
 0 & -\frac{1}{2} \left(k_2^2+4\right) & 0 & 0 & 0 & 0 & \frac{k_2}{2} & 0 & 0 & 0 \\
 4 \mu  & 0 & 3 \mu  & 2 \mu  k_2 & 0 & 0 & 0 & 0 & 0 & 0 \\
 0 & \frac{3}{2} \mu  \left(k_2^2-4 k_3\right) & 0 & 0 & 0 & 0 & -\frac{3 \mu  k_2}{2} & 0 & 0 & 0 \\
 0 & 0 & 0 & 0 & 0 & 0 & -2 \nu  & 0 & 0 & 0 \\
 4 \nu  & 0 & 3 \nu  & 2 \nu  k_2 & 0 & 0 & 0 & 0 & 0 & 0 \\
 0 & 0 & 0 & 0 & 0 & 0 & -3 \mu  & 0 & 0 & 0 \\
 0 & 0 & 0 & k_2 \left(k_3+1\right) & 0 & 0 & 0 & 0 & 0 & 0 \\
 0 & \frac{1}{2} \nu  \left(k_2^2-4 k_3\right) & 0 & 0 & 0 & 0 & -\frac{\nu  k_2}{2} & 0 & 0 & 0 \\
 0 & 0 & 0 & 0 & k_2 \left(k_3+1\right) & 0 & 0 & 0 & 0 & -k_3 \\
 0 & -k_2 \left(k_3+1\right) & 0 & 0 & 0 & 0 & k_3 & 0 & 0 & 0 \\
 0 & 0 & 0 & 0 & 0 & \frac{1}{2} \left(k_2^2-4 k_3\right) & 0 & 0 & 0 & 0 \\
\end{array}
\right)}.
\end{equation}

Очевидно, что $\mathrm{rank}~W = \mathrm{rank}~A + \mathrm{rank}~B$. Так как матрица $A$ является диагональной, то $\mathrm{rank}~W = 14 + \mathrm{rank}~B$. В таком случае, найдем ранг матрицы $B$, приведя её к ступенчатому виду. 

Из~(\ref{eq:aux_matrix}) видно, что в матрице $B$ существуют независимые столбцы с заведомо ненулевыми элементами. Это, во-первых, столбцы 2 и 5 с элементами вида $k_2^2 + 4$, и, во-вторых, столбцы 7 и 1 (или 3) с элементами $\mu$ и $\nu$. Таким образом, ранг матрицы $B$ не меньше четыркх. Пусть далее $\mu \ne 0$ (случай $\nu \ne 0$ рассматривается по аналогии). 

Далее, придется рассмотреть два случая "--- когда $k_2 = 0$ и $k_2 \ne 0$. В первом случае, матрица $B$ приводится к виду
\begin{equation*} 
\left(
\begin{array}{cccccccccc}
 2 & 0 & 0 & 0 & 0 & 0 & 0 & 0 & 0 & 0 \\
 0 & -2 & 0 & 0 & 0 & 0 & 0 & 0 & 0 & 0 \\
 0 & 0 & 3 \mu  & 0 & 0 & 0 & 0 & 0 & 0 & 4 \mu  \\
 0 & 0 & 0 & -3 \mu  & 0 & 0 & 0 & 0 & 0 & 0 \\
 0 & 0 & 0 & 0 & -2 k_3 & 0 & 0 & 0 & 0 & 0 \\
 0 & 0 & 0 & 0 & 0 & -k_3 & 0 & 0 & 0 & 0 \\
 0 & 0 & 0 & 0 & 0 & 0 & -2 \left(k_3-1\right) & 0 & 0 & 0 \\
   &   &   &   &   & \hdots &   &   &   &   \\
 0 & 0 & 0 & 0 & 0 & 0 & 0 & 0 & 0 & 0 \\
\end{array}
\right).
\end{equation*} 
Очевидно, что ранг матрицы $B$ будет равен 5 при $k_3 = 0$, 6 при $k_3 = 1$ и 7 в ином случае. Таким образом, ранг матрицы $W$ будет равен 19, 20 и 21 соответственно, а размерность соответствующей группы преобразований, как видно из~(\ref{eq:dim_rank}), равна 5, 4 и 3.

В случае, когда $k_2 \ne 0$, матрица $B$ приобретает следующий вид:
\begin{equation*}
{\small\left(
\begin{array}{cccccccccc}
 \frac{1}{2} \left(k_2^2+4\right) & 0 & 0 & 0 & -\frac{k_2}{2} & 0 & 0 & 0 & 0 & 0 \\
 0 & \frac{1}{2} \left(-k_2^2-4\right) & 0 & \frac{k_2}{2} & 0 & 0 & 0 & 0 & 0 & 0 \\
 0 & 0 & 3 \mu  & 0 & 0 & 0 & 2 \mu  k_2 & 0 & 0 & 4 \mu  \\
 0 & 0 & 0 & -3 \mu  & 0 & 0 & 0 & 0 & 0 & 0 \\
 0 & 0 & 0 & 0 & \frac{k_2^2 - 4 k_3}{k_2^2+4} & 0 & 0 & 0 & 0 & 0 \\
 0 & 0 & 0 & 0 & 0 & \frac{1}{2} \left(k_2^2-4 k_3\right) & 0 & 0 & 0 & 0 \\
 0 & 0 & 0 & 0 & 0 & 0 & k_2 \left(k_3+1\right) & 0 & 0 & 0 \\
 0 & 0 & 0 & 0 & 0 & 0 & k_2^2-2 k_3+2 & 0 & 0 & 0 \\
   &   &   &   &   & \hdots &   &   &   &   \\
 0 & 0 & 0 & 0 & 0 & 0 & 0 & 0 & 0 & 0 \\
\end{array}
\right)}
\end{equation*}
Отсюда видно, что ранг матрицы $B$ равен 5 при $k_2 = 2\sqrt{k_3},~k_3 > 0$ и 7 в иных случаях, что означает размерности 5 и 3 у групп аффинных преобразований, сохраняющих соответствующие поверхности.

Производя аналогичные рассуждения для случая $k_1 = 0$, получаем следующую классификацию поверхностей по размерностях групп преобразований, сохраняющих эти поверхности: 
\begin{itemize}
\item $\dim G = 5$ соответствуют поверхности
\begin{gather}
	v x_2 = x_1^2 + x_2 (\mu x_2^2 + \nu y_2^2),\label{eq:surf_1} \\
	v x_2 = y_1^2 + x_2 (\mu x_2^2 + \nu y_2^2),\label{eq:surf_3} \\
	v x_2 = (x_1 + y_1 \sqrt{k_3})^2 + x_2 (\mu x_2^2 + \nu y_2^2),~(k_3 > 0);\label{eq:surf_2}
\end{gather}
\item $\dim G = 4$ соответствуют поверхности
\begin{gather}
	v x_2 = x_1^2 + y_1^2 + x_2 (\mu x_2^2 + \nu y_2^2);
\end{gather}
\item $\dim G = 3$ соответствуют поверхности
\begin{gather}
	v x_2 = x_1 y_1 + k_3 y_1 ^2 + x_2 (\mu x_2^2 + \nu y_2^2),(k_3 \ne 0)\label{eq:surf_4}\\
	v x_2 = x_1^2 + k_2 x_1 y_1 + k_3 y_1^2 + x_2 (\mu x_2^2 + \nu y_2^2),~(k_2 \ne 2\sqrt{k_3}, k_3 \ne 1),\label{eq:surf_5}
\end{gather}
\end{itemize}
Проанализируем полученные поверхности. 

Заметим, что поверхности~(\ref{eq:surf_1}) -- (\ref{eq:surf_2}) являются аффинно- эквивалентными. Действительно, произведя замену переменных $z_1 = -i/(\sqrt{k_3}+i)z^*_1$ в уравнении~(\ref{eq:surf_2}) и $z_1 = -i z^*_1$ в уравнении~(\ref{eq:surf_3}), получаем уравнение поверхности~(\ref{eq:surf_1}).

Также отметим, что поверхность~(\ref{eq:surf_4}) является частным случаем поверхности~(\ref{eq:surf_5}) с точностью до аффинного преобразования. 

Исходя их этих соображений, можно сформулировать следующую теорему:
\begin{theorem}\label{thm:special} размерность группы Ли аффинных преобразований, сохраняющих любую поверхность вида~(\ref{eq:special}), удовлетворяет неравенствам
$3 \le \dim G \le 5$, причем
\begin{itemize}
	\item $\dim G$ = 3 достигается на поверхностях вида
	\begin{equation}\label{eq:special_3}
		v x_2 = x_1^2 + A x_1 y_1 + B y_1^2 + x_2 (\mu x_2^2 + \nu y_2^2),~(A \ne 2\sqrt{B}, B \ne 1);
	\end{equation}
	\item $\dim G$ = 4 достигается на поверхностях вида
	\begin{equation}\label{eq:special_4}
		v x_2 = |z_1|^2 + x_2 (\mu x_2^2 + \nu y_2^2);
	\end{equation}
		\item $\dim G$ = 5 достигается на аффинно-однородных поверхностях следующего вида
	\begin{equation}\label{eq:special_5}
		v x_2 = x_1^2 + x_2 (\mu x_2^2 + \nu y_2^2).
	\end{equation}
\end{itemize}
\end{theorem}

Рассмотрим движения, которые сохраняют классы поверхностей, перечисленных в теореме~\ref{thm:special}. Для этого решим возникающие системы уравнений для каждого из случаев, чтобы получить инфинитезимальные преобразования, порождающие однопараметрические группы аффинных преобразований, сохраняющих поверхности из обозначенных случаев.

Для каждой поверхности из класса~(\ref{eq:special}) существуют три <<основных>> движения, сохраняющие эти поверхности. Инфинитезимальные преобразования, порождающие группу движений, имеют следующий вид:
\begin{gather*}
E_1 = 
\left(
\begin{array}{cccc}
 0 & 0 & 0 & 0 \\
 0 & 0 & 0 & 0 \\
 0 & 0 & 0 & 1 \\
\end{array}
\right),~
E_2 = \left(
\begin{array}{cccc}
 3 & 0 & 0 & 0 \\
 0 & 2 & 0 & 0 \\
 0 & 0 & 4 & 0 \\
\end{array}
\right),\nonumber \\
E_3 = 
\left(
\begin{array}{cccc}
 0 & 0 & 0 & 0 \\
 0 & 0 & 0 & i \\
 0 & 2 \nu  & 0 & 0 \\
\end{array}
\right).
\end{gather*}
Рассмотрим соответствующие аффинные преобразования:
\begin{itemize}
\item \textit{сдвиг} по переменной $u$ ($E_1$). В данном случае получается следующая замена переменных: $z_1 = z_1^*$, $z_2 = z_2^*$, $w = w + i\cdot t$. Сохранение поверхности очевидно, так как она не зависит от переменной $u$;

\item \textit{масштабирование} ($E_2$). Преобразование имеет вид: $z_1 = e^{3t} z_1^*$, $z_2 = e^{2t} z_2^*$, $w = e^{4t} w^*$. Убедимся, что поверхность действительно сохраняется:
\begin{equation*}
v^* x_2^* \cdot e^{6t} = e^{6t} \cdot\left[k_1 (x_1^*)^2 + k_2 x_1^* y_1^* + k_3 (y_1^*)^2\right] + e^{6t}\cdot x_2^* \cdot\left[\mu (x_2^*)^2 + \nu (y_2^*)^2\right]
\end{equation*}
Множитель $e^{6t}$ является общим для обеих частей уравнения. После деления на него, получается исходная поверхность~(\ref{eq:special});

\item \textit{поворот со сдвигом (<<скользящий поворот>>)} ($E_3$). Замена переменных: $z_1 = z_1^*$, $z_2 = z_2^* + i\cdot t$, $w = w^* + 2t\nu z_2^* + i\cdot t^2 \nu$. Применим данное преобразование к произвольной поверхности~(\ref{eq:special}):
\begin{equation*}
\left(v^* + t^2 \nu + 2 t \nu y_2\right) x_2^* = k_1 (x_1^*)^2 + k_2 x_1^* y_1^* + k_3 (y_1^*)^2 + x_2^* \left[\mu (x_2^*)^2 + \nu \left(y_2^* + i t\right)^2\right]
\end{equation*}
Перегруппируем слагаемые следующим образом:
\begin{multline*}
\left(v^* x_2^*\right) + \nu t^2 x_2 + 2\nu t x_2 y_2 = k_1 (x_1^*)^2 + k_2 x_1^* y_1^* + k_3 (y_1^*)^2 + \\
+ x_2^* \left[\mu (x_2^*)^2 + \nu \left(y_2^*\right)^2\right] + \nu t^2 x_2 + 2\nu t x_2 y_2.
\end{multline*}
Видно, что член $\nu t^2 x_2 + 2\nu t x_2 y_2$ присутствует в обеих частях уравнения и может быть сокращен, после чего получается исходное уравнение.
\end{itemize}

Для поверхностей типа~(\ref{eq:special_4}) добавляется еще один вид движений "--- повороты в плоскости $z_1$. Соответствующее инфинитезимальное преобразование и порождаемое аффинное преобразование:
\begin{equation*}
E_4 = \left(
\begin{array}{cccc}
 i & 0 & 0 & 0 \\
 0 & 0 & 0 & 0 \\
 0 & 0 & 0 & 0 \\
\end{array}
\right),~
\left(
\begin{array}{c}
 z_1 \\
 z_2 \\
 w \\
\end{array}
\right) = 
\left(
\begin{array}{c}
 e^{it} \cdot z_1^* \\
 z_2^* \\
 w^* \\
\end{array}
\right).
\end{equation*}
Факт сохранения поверхности этим преобразованием довольно очевиден: так как $|e^{it}| = 1$, уравнение поверхности~(\ref{eq:special_4}) останется неизменным.

Поверхности~(\ref{eq:special_5}), в дополнение к трем основным, имеют еще два типа движений, которым соответствуют следующие инфинитезимальные преобразования:
\begin{equation*}
E_5 = \left(
\begin{array}{cccc}
 0 & 0 & 0 & i \\
 0 & 0 & 0 & 0 \\
 0 & 0 & 0 & 0 \\
\end{array}
\right),~
E_6 = \left(
\begin{array}{cccc}
 0 & 1 & 0 & 0 \\
 0 & 0 & 0 & 0 \\
 2i & 0 & 0 & 0 \\
\end{array}
\right).
\end{equation*}
Порождаемые аффинные преобразования:
\begin{itemize}
\item \textit{сдвиг} по переменной $y_1$. Замена переменных "--- $z_1 = z_1^* + i t$, $z_2 = z_2^*$, $w = w^*$. Факт сохранения поверхности этим преобразованием очевиден, так как уравнение поверхности~(\ref{eq:special_5}) не зависит от переменной $y_1$.

\item \textit{поворот}:
\begin{equation*}
\left(
\begin{array}{c}
 z_1 \\
 z_2 \\
 w \\
\end{array}
\right) = 
\left(
\begin{array}{c}
 z_1^* + t \cdot z_2^* \\
 z_2^*\\
 2 i t z_1^* + i t^2 z_2^* + w^*\\
\end{array}
\right).
\end{equation*}
Проверим сохранение поверхности~(\ref{eq:special_5}) этим преобразованием:
\begin{equation*}
\left(t^2 x_2^* +2 t x_1^* + v^*\right) x_2^* = \left(x_1^* + t x_2^*\right){}^2 + x_2^* \left[\mu  (x_2^*)^2+\nu  (y_2^*)^2\right]
\end{equation*}
Раскрывая скобки и перегруппировывая слагаемые:
\begin{equation*}
v^* x_2^* + 2 t^2 (x_2^*)^2 + 2 t x_1^* x_2^* = (x_1^*)^2 + x_2^* \left[\mu  (x_2^*)^2+\nu  (y_2^*)^2\right]+ 2t^2 (x_2^*)^2 + 2 t x_1^* x_2^* 
\end{equation*}
Слагаемое $2t^2 (x_2^*)^2 + 2 t x_1^* x_2^*$, которое присутствует в обеих частях, может быть сокращено. В таком случае, получается уравнение~(\ref{eq:special_5}).
\end{itemize}

В рамках данной работы было также произведено расширение класса аффинно-однородных поверхностей~(\ref{eq:homogenous}) за счет расширения кубической формы в правой части:
\begin{theorem}\label{special_homogenous}
для тройки коэффициентов $\mu$,$\nu$,$\lambda$, одновременно не равных нулю, поверхность
\begin{equation}\label{eq:special_5_homogenous}
v x_2 = x_1^2 + x_2 (\mu x_2 ^2 + \lambda x_2 y_2 + \nu y_2 ^2)
\end{equation}
является аффинно-однородной, а размерность группы Ли аффинных преобразований, сохраняющих такую поверхность, равна 5.
\end{theorem}
\begin{proof}

Рассмотрим уравнение поверхности~(\ref{eq:special_5_homogenous}) в точке $(0, 1, i\mu)$. Для этого, совершим следующую замену переменных: $z_1 = z^*_1$, $z_2 = z^*_2 + 1$, $w = w^* + i\cdot \mu$. Тогда уравнение принимает следующий вид (звездочки опущены для наглядности):
\begin{equation*}
v \left(x_1+1\right)=x_1^2 + \left(x_2+1\right) \left(\mu  x_2^2+\nu  y_2^2\right) + \left(x_2+1\right) \left(2 \mu  x_2+\lambda  y_2\right).
\end{equation*}

Далее применим поворот $z_1 = z^*_1$, $z_2 = z^*_2 $, $w = w^* + (\lambda + 2 \mu i) z^*_2$, чтобы привести уравнение к следующему виду:
\begin{equation*}
v (x_2 + 1) = x_1^2 + (x_2 + 1) (\mu x_2 ^2 + \lambda x_2 y_2 + \nu y_2 ^2).
\end{equation*}

Теперь покажем, что группа аффинных преобразований, сохраняющих данную поверхность, имеет размерность 5. Для этого, построим систему уравнений на коэффициенты инфинитезимальных преобразований по алгоритму, описанному в начале данной главы, по аналогии с тем, что уже проделано для случая~(\ref{eq:special}). Отметим лишь, что различие будет в возникающем знаменателе (он будет равен $(1 + x_2)^2$). 

В результате возникает система из 61 уравнения относительно 24 неизвестных. Схема основной матрицы этой системы представлена в приложении Г. Из схемы видно, что в матрице имеется большое число заведомо ненулевых элементов, которые располагаются в 17 разных столбцах. В таком случае, приведем эту матрицу к ступенчатому виду. 

Матрицу системы $W$ можно записать в квазитреугольном виде:
\begin{equation*}
W \sim
\begin{pmatrix}
A & N \\
0 & B \\
\hdots & \hdots  \\
0 & 0
\end{pmatrix},
\end{equation*}
где матрица $A$ "--- диагональная матрица размерами $17\times17$, $N\in \mathbb{R}^{17\times7}$, $B \in \mathbb{R}^{26\times7}$ и имеет вид:
\begin{equation}\label{eq:matrix_big}
\left(
\begin{array}{ccccccc}
 -2 \nu  & 0 & 0 & 0 & 0 & 0 & 0 \\
 6 \lambda  & 0 & 0 & 0 & 0 & 0 & 0 \\
 -12 \mu  & 0 & 0 & 0 & 0 & 0 & 0 \\
 4 \lambda  & 0 & 0 & 0 & 0 & 0 & 0 \\
 -9 \mu  & 0 & 0 & 0 & 0 & 0 & 0 \\
 0 & 0 & 0 & -2 \mu  & 0 & 0 & \mu  \\
 -\lambda  & 0 & 0 & 0 & 0 & 0 & 0 \\
 0 & 0 & 0 & -2 \lambda  & 0 & 0 & \lambda  \\
 -3 \mu  & 0 & 0 & 0 & 0 & 0 & 0 \\
 -2 \nu  & 0 & 0 & 0 & 0 & 0 & 0 \\
 -2 \lambda  & 0 & 0 & 0 & 0 & 0 & 0 \\
 0 & 0 & 0 & -2 \nu  & 0 & 0 & \nu  \\
 -\nu  & 0 & 0 & 0 & 0 & 0 & 0 \\
 -6 \nu  & 0 & 0 & 0 & 0 & 0 & 0 \\
 6 \lambda  & 0 & 0 & 0 & 0 & 0 & 0 \\
 0 & 0 & 0 & -6 \nu  & 0 & 0 & 3 \nu  \\
 0 & 0 & 0 & 12 \lambda  & 0 & 0 & -6 \lambda  \\
 0 & 0 & 0 & -20 \mu  & 0 & 0 & 10 \mu  \\
 8 \lambda  & 0 & 0 & 0 & 0 & 0 & 0 \\
 0 & 0 & 0 & -6 \nu  & 0 & 0 & 3 \nu  \\
 0 & 0 & 0 & 16 \lambda  & 0 & 0 & -8 \lambda  \\
 0 & 0 & 0 & -30 \mu  & 0 & 0 & 15 \mu  \\
 3 \lambda  & 0 & 0 & 0 & 0 & 0 & 0 \\
 0 & 0 & 0 & -2 \nu  & 0 & 0 & \nu  \\
 0 & 0 & 0 & 6 \lambda  & 0 & 0 & -3 \lambda  \\
 0 & 0 & 0 & -12 \mu  & 0 & 0 & 6 \mu  \\
\end{array}
\right).
\end{equation}

Видно, что в матрице~(\ref{eq:matrix_big}) столбцы 4 и 7 являются линейно зависимыми. В таком случае, ранг матрицы $B$ равен двум, а размерность группы преобразований, сохраняющих поверхность~(\ref{eq:special_5_homogenous}) равна пяти (см.~(\ref{eq:dim_rank})).

Рассмотрим теперь независимые инфинитезимальные преобразования, которые получаются после решения системы линейных уравнений:
\begin{equation}\label{eq:matrices} 
\begin{gathered}
E_1 = \left(
\begin{array}{cccc}
 0 & 1 & 0 & 1 \\
 0 & 0 & 0 & 0 \\
 2i & 0 & 0 & 0 
\end{array}
\right),~
E_2 = \left(
\begin{array}{cccc}
 \frac{3}{2} & 0 & 0 & 0 \\
 0 & 1 & 0 & 1 \\
 0 & \lambda +2 i \mu & 1 & 0 
\end{array}
\right),
E_3 = \left(
\begin{array}{cccc}
 0 & 0 & 0 & i \\
 0 & 0 & 0 & 0 \\
 0 & 0 & 0 & 0 
\end{array}
\right),\\
E_4 = \left(
\begin{array}{cccc}
 0 & 0 & 0 & 0 \\
 0 & 0 & 0 & i \\
 0 & i \lambda +2 \nu  & 0 & 0 
\end{array}
\right),~
E_5 = \left(
\begin{array}{cccc}
 0 & 0 & 0 & 0 \\
 0 & 0 & 0 & 0 \\
 0 & 0 & 0 & 1 
\end{array}
\right)
\end{gathered}
\end{equation}

Следует обратить внимание на четвертый столбец в матрицах~(\ref{eq:matrices}). Как известно, четвертый столбец в матрице~(\ref{eq:infinitesimal_matrix}) отвечает за сдвиговую часть в порождаемой однопараметрической группе аффинных преобразований~\cite{lie}. Благодаря произведенным преобразованиям, можно увидеть, что направления движений, задаваемые инфинитезимальными преобразованиями $E_1$ -- $E_5$, позволяют сдвигаться под действием порождаемой группы в любом направлении вдоль поверхности~(\ref{eq:special_5_homogenous}). Таким образом, группа аффинных преобразований, сохраняющая эту поверхность действует транзитивно, а значит, поверхность является аффинно-однородной.

\end{proof}
%%%%%%%
\subsection{Произвольная поверхность}
В общем случае квадратичную форму $Q(x_1, y_1, x_2, y_2)$ можно записать в явном виде:
\begin{align*}
Q(x_1, y_1, x_2, y_2) &= k_1 x_1^2 + k_2 x_2 x_1 + k_3 x_2^2 + k_4 x_1 y_1 + k_5 x_2 y_1 \\
&+ k_6 y_1^2 + k_7 x_1 y_2 + k_8 x_2 y_2 + k_9 y_1 y_2 + k_{10} y_2^2
\end{align*}

Применяя метод, указанный в предыдущем пункте, получим линейную однородную систему из 83 уравнений относительно 24 неизвестных. Наиболее простые уравнения в системе имеют вид
\begin{equation}
\begin{matrix}
 k_1 \beta _{3,2}=0 &  k_1 \beta _{3,1}=0 &  k_1 \sigma _{2,1}=0 \\
 k_6 \beta _{3,2}=0 & k_6 \beta _{3,1}=0 & k_6 \sigma _{2,1}=0 \\
 k_4 \beta _{3,1}=0 &  k_4 \sigma _{2,1}=0 \\
 k_7 \sigma _{2,1}=0 & k_7 \beta _{3,1}=0 \\
 k_9 \beta _{3,1}=0 & k_9 \sigma _{2,1}=0 \\
 k_{10} \beta _{3,2}=0 &  & 
\end{matrix}
\end{equation}
Отсюда можно выделить два случая:
\begin{enumerate}
	\item Все $k_j$ равны нулю: $k_1 = k_4 = k_6 = k_7 = k_9 = k_{10} = 0$.
	\item Хотя бы один $k_j$ из указанного набора не равен нулю.
\end{enumerate}
В первом случае поверхность приобретает следующий вид:
\begin{equation*}
v x_2 = k_2 x_2 x_1 + k_3 x_2^2 + k_5 x_2 y_1 + k_8 x_2 y_2 + x_2 (\mu x_2^2 + \nu y_2^2).
\end{equation*}
Это один из тривиальных частных случаев, обозначенных в разделе 1 данной работы "--- квадратичная форма делится на $x_2$. Таким образом, можно перейти сразу к рассмотрению случая 2.

По аналогии с рассуждениями для частного случая, предположим, что $k_1 \ne 0 $. В таком случае, за счёт аффинных преобразований, параметр $k_1$ можно сделать равным единицей. Приведём основную матрицу системы уравнений $W$ к ступенчатому виду элементарными преобразованиями. Матрица принимает вид (по аналогии с выкладками выше): 
\begin{equation*}
W \sim
\begin{pmatrix}
A & N \\
0 & B \\
\hdots & \hdots  \\
0 & 0
\end{pmatrix},
\end{equation*}
где матрица $A$ "--- диагональная матрица размерами $18\times18$, $N\in \mathbb{R}^{18\times6}$, $B \in \mathbb{R}^{16\times6}$ и имеет вид:
\begin{equation}\label{eq:matrix_general}
{\left(
\begin{array}{cccccc}
 k_4^2-2 k_6+2 & 0 & 0 & 0 & 0 & 0 \\
 k_4 \left(k_6+1\right) & 0 & 0 & 0 & 0 & 0 \\
 \frac{1}{3\left(k_4^2+4\right)}f_1 & 0 & \frac{1}{3 \left(k_4^2+4\right)}f_3 & 0 & 0 & -\frac{4
   k_6-k_4^2}{k_4^2+4} \\
 0 & \frac{1}{2} \left(k_4^2-4 k_6\right) & 0 & \frac{1}{2} \left(k_4 k_7-2 k_9\right) & 0 & 0 \\
 \frac{1}{3\left(k_4^2+4\right)}f_2 & 0 & \frac{1}{3 \left(k_4^2+4\right)}f_4 & 0 & 0 & -\frac{2k_9-k_4 k_7}{k_4^2+4} \\
 0 & \frac{1}{2} \left(k_4 k_7-2 k_9\right) & 0 & \frac{1}{2} \left(k_7^2-4 k_{10}\right) & 0 & 0 \\
  &  & \hdots &  &  & \\
 0 & 0 & 0 & 0 & 0 & 0
\end{array}
\right)},
\end{equation}
а коэффициенты $f_1$ -- $f_4$ выглядят следующим образом:
\begin{align*}
f_1 &=
\begin{multlined}[t]
k_9 k_4^3-\left(3 k_5+4 k_6 k_7+k_7\right) k_4^2+2 \left(3 k_6+7\right) k_9 k_4+2 k_2 \left(k_4^3-4 k_4 k_6\right) + \\ 
+ 12 \left(k_5 k_6+k_7\right),
\end{multlined}\\
f_2 &= 
\begin{multlined}[t]
k_{10} k_4^3-2 k_7 k_9 k_4^2+2 k_2 \left(k_4 k_7-2 k_9\right) k_4+\left(-4 k_7^2-3 k_5 k_7+3 k_9^2+4 k_{10}\right) k_4+ \\
+ 6 \left(k_5+k_7\right) k_9,
\end{multlined}\\
f_3 &= k_2 \left(k_4^2-4 k_6\right)-2 k_4 \left(k_6+1\right) k_7+\left(k_4^2+4\right) k_9, \\ 
f_4 &= k_2 \left(k_4 k_7-2 k_9\right)-k_7 \left(2 k_7+k_4 k_9\right)+2 \left(k_4^2+4\right) k_{10}.
\end{align*}

Ранг матрицы $W$ в таком случае будет равен $18 + \mathrm{rank}~B$. Тогда изучим ранг матрицы $B$. 

Из~(\ref{eq:matrix_general}) видно, что ранг матрицы $B$ изменяется от принимает значения от 1 до 5. Это означает, что размерность группы аффинных преобразований, сохраняющих поверхности из семейства~(\ref{eq:initial}) изменяется от 1 до 5. Примеры значений параметров $k_2,\hdots,k_{10}$ для каждой из размерностей:
\begin{align*}
\dim G = 1,&\qquad k_4 = k_6 = k_{10} = 1,~k_2 = k_3 = k_5 = 0,~k_7 = \hdots = k_9 = 0 \\
\dim G = 2,&\qquad k_6 = k_{10} = 1,~k_2 = \hdots = k_5 = 0, k_7 = \hdots = k_9 = 0 \\
\dim G = 3,&\qquad k_4 = k_6 = 1,~k_2 = \hdots = k_5 = 0, k_7 = \hdots = k_{10} = 0 \\
\dim G = 4,&\qquad k_6 = 1,~k_2 = \hdots = k_5 = 0,~k_7 = \hdots = k_{10} = 0 \\
\dim G = 5,&\qquad k_2 = \hdots = k_{10} = 0 
\end{align*}

Проводя аналогичные рассуждения для остальных параметров, можно сформулировать следующую теорему:

\begin{theorem} Размерность группы Ли аффинных преобразований, сохраняющих любую поверхность из семейства~(\ref{eq:initial}), удовлетворяет неравенству $1 \le \dim G \le 5$, и каждая из размерностей достижима.
\end{theorem}
\begin{proof}
Первая часть доказательства, связанная с оценкой размерности группы, представлена выше. В таком случае, покажем, что каждая из этих размерностей достижима. 

В теореме~\ref{thm:special} данной работы приведены поверхности, для которых размерности группы изменяются от 3 до 5. Остается доказать достижимость размерностей 1 и 2. Рассмотрим следующие поверхности, которые получились после рассмотрения рангов матрицы~(\ref{eq:matrix_general}):
\begin{align}
\dim G = 1, &\qquad v x_2 = {|z_1|}^2 + x_1 y_1 + y_2 ^2 + x_2 {|z_2|}^2, \label{eq:general_1}\\
\dim G = 2, &\qquad v x_2 = {|z_1|}^2 + y_2^2 + x_2 {|z_2|}^2.\label{eq:general_2}
\end{align}

С помощью программы на языке {\ttfamily Wolfram Language}, представленной в приложении, были получены группы преобразований для обеих заявленных поверхностей. Первую поверхность сохраняет сдвиг по переменной $u$. Для второй поверхности добавляется еще одно движение "--- повороты в плоскости $z_1$. 
\end{proof}

\end{document}