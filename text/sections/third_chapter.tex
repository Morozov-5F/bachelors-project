% !TEX root = ../main.tex
\documentclass[../main.tex]{subfiles}
\begin{document}

В данном разделе будет описана процедура определения размерностей групп аффинных преобразований для семейства поверхностей~(\ref{eq:initial}).
\subsection{Общая схема}
Основываясь на сведениях из раздела 2, можно сформулировать следующий алгоритм для нахождения размерности групп аффинных преобразований, сохраняющих поверхности из семейства~(\ref{eq:initial}):
\begin{enumerate}
\item Применить аффинное преобразование~(\ref{eq:affine_transform}) к уравнению поверхности
\item Продифференцировать полученное выражение по параметру группы $t$ в точке $t = 0$.
\item Разрешить уравнение поверхности относительно переменной $v$ и подставить в выражение и предыдущего шага и умножить на возникающий общий знаменатель ($x_2^2$).
\item Составить однородную систему линейных относительно элементов инфинитезимального преобразования уравнений и выписать матрицу этой системы.
\item Размерность группы Ли аффинных преобразований $G$, сохраняющих поверхность из исследуемого семейства, равна $\dim G = 24 - \mathrm{rank}~W$, где $W$ "--- полученная матрица системы.
\end{enumerate}

%%%%%%
\subsection{Частный случай семейства}
В связи со сложностью рассмотрения общего случая, был рассмотрен частный случай семейства поверхностей~(\ref{eq:initial}), в котором квадратичная форма $Q$ не зависит от переменных $x_2$ и $y_2$. В таком случае, семейство может быть описано следующим образом:
\begin{equation}\label{eq:special}
v x_2 = k_1 x_1^2 + k_2 x_1 y_1 + k_3 x_2^2 + x_2 (\mu x_2^2 + \nu y_2^2),
\end{equation}

Следует заметить, что за счёт применения невырожденных аффинных преобразований, можно фактически сократить число параметров $k_1, k_2$ и $k_3$ до одного.

Применим описанный метод для этого класса поверхностей.

\begin{enumerate}
\item Применим аффинное преобразование к определяющей функции поверхности из семейства~(\ref{eq:special}):
\begin{multline}\label{eq:special_transformed}
F_t(\Phi) = l_6(x_1, y_1, x_2, y_2, u, v) \cdot l_3(x_1, y_1, x_2, y_2, u, v) + k_1 l_1(x_1, y_1, x_2, y_2, u, v){}^2 + \\ + k_2 l_1(x_1, y_1, x_2, y_2, u, v)\cdot l_2(x_1, y_1, x_2, y_2, u, v) + k_3 l_2(x_1, y_1, x_2, y_2, u, v){}^2 - \\ 
- l_3(x_1, y_1, x_2, y_2, u, v) \left[\mu l_3(x_1, y_1, x_2, y_2, u, v){}^2+\nu  l_4(x_1, y_1, x_2, y_2, u, v){}^2\right]
\end{multline}

\item Производная~(\ref{eq:special_transformed}) по $t$ в точке $t = 0$:
\begin{multline}\label{eq:special_diff}
\left.\frac{\mathrm d}{\mathrm{d}t} F_t\left(\Phi\right)\right|_{t = 0} = v l_3^{\prime}(x_1, y_1, x_2, y_2, u, v) + x_2 l_6^{\prime}(x_1, y_1, x_2, y_2, u, v) + \\
+ 2 k_1 x_1 l_1^{\prime}(x_1, y_1, x_2, y_2, u, v) + k_2 y_1 l_2^{\prime}(x_1, y_1, x_2, y_2, u, v) + \\
+ k_2 x_1 l_2^{\prime}(x_1, y_1, x_2, y_2, u, v) +2 k_3 y_1 l_2^{\prime}(x_1, y_1, x_2, y_2, u, v) - \left(\mu x_2^2+\nu  y_2^2\right) \times \\
\times l_3^{\prime}(x_1, y_1, x_2, y_2, u, v) - x_2 \left[2 \mu  x_2 l_3^{\prime}(x_1, y_1, x_2, y_2, u, v) + \right. \\
\left. +2 \nu  y_2 l_4^{\prime}(x_1, y_1, x_2, y_2, u, v)\right]
\end{multline}

\item Разрешим уравнение~(\ref{eq:special}) относительно $v$:
$$
v = \frac{k_1 x_1^2 + k_2 x_1 y_1 + k_3 x_2^2}{x_2} +  \mu x_2^2 + \nu y_2^2
$$
и подставим его в~(\ref{eq:special_diff}), умножая на $x_2^2$. В данном многочлене имеется 57 мономов и 101 слагаемое. Приведем подобные и выпишем по одному примеру одночленов для каждой из степеней:
\begin{align*}
\text{степень 6}&:\ x_1 x_2^2 y_2 \left(k_2 \alpha _{2,1}-2 k_1 \alpha _{2,2}\right) \\
\text{степень 5}&:\ x_1 x_2^4 \left(-2 \mu  \beta _{1,1}+k_2 \mu  \alpha _{3,1}-2 k_1 \mu  \alpha _{3,2}\right) \\ 
\text{степень 4}&:\ x_1 x_2 y_1^2 \left(k_2^2 \alpha _{3,2}-3 k_3 k_2 \alpha _{3,1}+2 k_1 k_3 \alpha _{3,2}+k_2 \beta _{1,2}-k_3 \beta _{1,1}\right) \\
\text{степень 3}&:\ x_2^3 \sigma _{3,2}
\end{align*}
Видно, что есть как довольно простые (степень 3), так и довольно сложные (степень 4) коэффициенты при одночленах.

\item Составим однородную систему линейных уравнений на основе многочлена, полученного в предыдущем пункт. Получается система из 57 уравнений относительно 24 неизвестных. Схему основной матрицы этой системы можно посмотреть в приложении 3. 

В качестве примеров, приведем некоторые уравнения из этой системы:
\begin{align*}
k_2^2 \left(-\alpha _{3,1}\right)+3 k_1 k_2 \alpha _{3,2}-2 k_1 k_3 \alpha _{3,1}-k_2 \beta _{1,1}+k_1 \beta _{1,2} &= 0 \\
2 k_2 \alpha _{1,1}-2 k_1 \alpha _{1,2}+2 k_3 \alpha _{1,2}-k_2 \beta _{2,1}-k_2 \gamma _{3,1} &=0 \\
\left(-k_2^2-2 k_1 k_3\right) \beta _{3,2} &=0 \\ 
\gamma_{3,2} &= 0\\ 
\end{align*}

\item Для изучения ранга этой матрицы, приведем её элементарными преобразованиями к ступенчатому виду.

\end{enumerate}



\begin{theorem} размерность группы Ли аффинных преобразований, сохраняющих любую поверхность вида~(\ref{eq:special}), удовлетворяет неравенствам
$3 \le \dim G \le 5$, причем
\begin{itemize}
	\item $\dim G$ = 3 достигается на поверхностях вида
	\begin{equation}\label{eq:special_3}
		v x_2 = k x_1^2 + y_1^2 + x_2 (\mu x_2^2 + \nu y_2^2),\ k \ne 0, \ k \ne 1;
	\end{equation}
	\item $\dim G$ = 4 достигается на поверхностях вида
	\begin{equation}\label{eq:special_4}
		v x_2 = |z_1|^2 + x_2 (\mu x_2^2 + \nu y_2^2);
	\end{equation}
		\item $\dim G$ = 5 достигается на аффинно-однородных поверхностях следующего вида
	\begin{equation}\label{eq:special_5}
		v x_2 = x_1^2 + x_2 (\mu x_2^2 + \nu y_2^2).
	\end{equation}
\end{itemize}
\end{theorem}

%%%%%%%
\subsection{Произвольная поверхность}
В общем случае, квадратичную форму $Q(x_1, y_1, x_2, y_2)$ можно записать в явном виде:
\begin{align*}
Q(x_1, y_1, x_2, y_2) &= k_1 x_1^2 + k_2 x_2 x_1 + k_3 x_2^2 + k_4 x_1 y_1 + k_5 x_2 y_1 \\
&+ k_6 y_1^2 + k_7 x_1 y_2 + k_8 x_2 y_2 + k_9 y_1 y_2 + k_{10} y_2^2
\end{align*}

Применяя метод, указанный в предыдущем пункте, получим линейную однородную систему из 83 уравнений относительно 24 неизвестных, причем переменная $\sigma_{3,1}$ является <<фиктивной>> "--- в связи с тем, что   Наиболее простые уравнения в системе имеют вид
\begin{equation}
\begin{matrix}
 k_1 \beta _{3,2}=0 &  k_1 \beta _{3,1}=0 &  k_1 \sigma _{2,1}=0 \\
 k_6 \beta _{3,2}=0 & k_6 \beta _{3,1}=0 & k_6 \sigma _{2,1}=0 \\
 k_4 \beta _{3,1}=0 &  k_4 \sigma _{2,1}=0 \\
 k_7 \sigma _{2,1}=0 & k_7 \beta _{3,1}=0 \\
 k_9 \beta _{3,1}=0 & k_9 \sigma _{2,1}=0 \\
 k_{10} \beta _{3,2}=0
\end{matrix}
\end{equation}
Отсюда можно выделить два случая:
\begin{enumerate}
	\item Все $k_j$ равны нулю: $k_1 = k_4 = k_6 = k_7 = k_9 = k_{10} = 0$.
	\item Хотя бы один $k_j$ из указанного набора не равен нулю.
\end{enumerate}
В первом случае поверхность приобретает следующий вид:
\begin{equation*}
v x_2 = k_2 x_2 x_1 + k_3 x_2^2 + k_5 x_2 y_1 + k_8 x_2 y_2 + x_2 (\mu x_2^2 + \nu y_2^2).
\end{equation*}
Легко видеть, что данное уравнение можно поделить на $x_2$, что в итоге даст новую поверхность, которая является квадрикой. Рассмотрение поверхностей данного типа не представляет интереса в данной работе, поэтому можно сразу перейти к рассмотрению второго случая.

Предположим, что $k_1 \ne 0 $. В таком случае, приведем элементарными преобразованиями матрицу системы $W$ к ступенчатому виду.

Исходя их этих соображений можно сформулировать следующую теорему:

\begin{theorem} Размерность группы Ли аффинных преобразований, сохраняющих любую поверхность из семейства~(\ref{eq:initial}), удовлетворяет неравенству $1 \le \dim G \le 5$, и каждая из размерностей достижима.
\end{theorem}

\end{document}