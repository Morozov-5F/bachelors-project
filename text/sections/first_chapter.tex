% !TEX root = ../main.tex
\documentclass[../main.tex]{subfiles}
\begin{document}

Пусть дано семейство вещественных квадро-кубических гиперповерхностей в трехмерном комплексном пространстве:
\begin{equation}\label{eq:initial}
v x_2 = Q(x_1, y_1, x_2, y_2) + x_2 (\mu x_2^2 + \nu y_2^2),
\end{equation}
где $x_1 = \Re(z_1)$, $y_1 = \Im(z_1)$, $x_2 = \Re(z_2)$, $y_2 = \Im(z_2)$, $u = \Re(w)$, $v= \Im(w)$ "--- компоненты координат в трехмерном комплексном пространстве, $\mu$, $\nu$ "--- некоторые вещественные параметры, одновременно не равные нулю, а $Q(x_1, y_1, x_2, y_2)$ "--- некоторая квадратичная форма. Под симметрией любой поверхности из данного класса будем понимать любое аффинное преобразование, сохраняющее эту поверхность. В то же время, симметрией определяющей функции поверхности будем называть преобразование, сохраняющее это уравнение с точностью до ненулевого множителя:
\begin{equation}\label{eq:preservation}
G(\Phi(z, \overline{z}, v)) \equiv \psi(z, \overline{z}, w, \overline{w}) \cdot \Phi(z, \overline{z}, v),
\end{equation}
где $G$ "--- некоторое аффинное преобразование, $\Phi(z, \overline z, v)$ "--- определяющая функция поверхности, а $\psi(z, \overline z, w, \overline w)$ "--- ненулевая функция. Изучая симметрии в таком контексте, вполне естественно обсуждать группы аффинных преобразований, под действиями которых всякая поверхность из класса~(\ref{eq:initial}) сохраняется. Задача данной работы, в таком случае, заключается в изучении групп аффинных преобразований, сохраняющих конкретные поверхности из данного семейства, и размерностей таких групп. 

Заметим, что квадратичная форма $Q(x_1, y_1, x_2, y_2)$ не является совсем произвольной "--- первое требование состоит в том, что она должна быть невырождена  (это требование было унаследовано от задачи об однородности). Второе требование состоит в том, чтобы форма не делилась на $x_2$. Предположим, что эта форма делится на $x_2$, то есть имеет следующий вид:
\begin{equation*}
Q(x_1, y_1, x_2, y_2) = k_1 x_2 x_1 + k_2 x_2^2 + k_3 x_2 y_1 + k_4 x_2 y_2 + k_{5} y_2^2,
\end{equation*}
где $k_j$ "--- некоторые вещественные коэффициенты. В таком случае, можно поделить обе части~(\ref{eq:initial}) на $x_2$, чтобы получить <<обычную>> квадрику. Данный тип поверхностей не представляет интереса в рамках этой работы и изучаться не будет. 

Следует отметить, что можно изучать поставленную задачу с точностью до аффинных преобразований исходных объектов. На размерности и некоторые другие свойства групп это не повлияет.

TODO: нормальные формы
\end{document}