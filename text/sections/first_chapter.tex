% !TEX root = ../main.tex
\documentclass[../main.tex]{subfiles}
\begin{document}

Рассмотрим семейство вещественных <<квадро-кубических>> гиперповерхностей в трехмерном комплексном пространстве:
\begin{equation}\label{eq:main}
v x_2 + Q(x_1, y_1, x_2, y_2) = T(x_1, y_1, x_2, y_2)
\end{equation}
где $x_1 = \Re(z_1)$, $y_1 = \Im(z_1)$, $x_2 = \Re(z_2)$, $y_2 = \Im(z_2)$, $u = \Re(w)$, $v= \Im(w)$ "--- компоненты~~~~ координат ~~~в ~~~~трехмерном ~~~~комплексном ~~~~пространстве, \\$Q(x_1, y_1, x_2, y_2)$ "--- некоторая квадратичная, а $T(x_1, y_1, x_2, y_2)$ "--- некоторая кубическая формы. Под симметрией любой поверхности из данного класса будем понимать любое аффинное преобразование, сохраняющее эту поверхность. В то же время, симметрией определяющей функции поверхности будем называть преобразование, сохраняющее это уравнение с точностью до ненулевого множителя:
\begin{equation}\label{eq:preservation}
G(\Phi(z, \overline{z}, v)) \equiv \psi(z, \overline{z}, w, \overline{w}) \cdot \Phi(z, \overline{z}, v),
\end{equation}
где $G$ "--- некоторое аффинное преобразование, $\Phi(z, \overline z, v)$ "--- определяющая функция поверхности, а $\psi(z, \overline z, w, \overline w)$ "--- ненулевая функция. Изучая симметрии в таком контексте, вполне естественно обсуждать группы аффинных преобразований, под действиями которых всякая поверхность из класса~(\ref{eq:main}) сохраняется. Задача данной работы, в таком случае, заключается в изучении групп аффинных преобразований, сохраняющих конкретные поверхности из данного семейства, и размерностей таких групп.

Следует сразу отметить, что в классе~(\ref{eq:main}) присутствуют частные случаи поверхностей, являющиеся тривиальными. Таковыми являются, например, поверхности, в которых формы $Q(x_1, y_1, x_2, y_2)$ и $T(x_1, y_1, x_2, y_2)$ делятся на $x_2$. Деление уравнения~(\ref{eq:main}) на эту переменную понижает его степень и поверхность выходит из класса кубических. Такие случаи в данной работе рассматриваться не будут.

В задачах, связанных с однородностью, отдельно проверяется свойство невырожденности вещественно-аналитических поверхностей в смысле Леви. Для этого уравнение поверхности необходимо разрешить относительно одной из вещественных переменных, разложить получившуюся функцию в степенной ряд, и аффинными преобразованиями свести к виду
\begin{equation}
v = H(z, \overline z) + Q(z) + \overline{Q(z)} + \sum_{k \ge 3} F_k(z, \overline z, u),
\end{equation}
где $H(z, \overline z)$ "--- эрмитова, $Q(z)$ "--- квадратичная формы, $F_k(z, \overline z, u)$ "--- компонента однородности веса $k$. В таком случае, поверхность называется вырожденной по Леви, если эрмитова форма $H$ является вырожденной.

Заметим, что можно изучать поставленную задачу с точностью до аффинных преобразований исходных объектов. На размерности и некоторые другие свойства групп это не повлияет.

В рамках данной работы была предпринята попытка изучения задачи в общей постановке~(\ref{eq:main}), и был разработан метод для её решения. Однако, как будет видно далее, объекты, возникшие в ходе решения, оказались чересчур громоздкими. Поэтому, в настоящем исследовании изучается более узкий класс поверхностей из данного семейства:
\begin{equation}\label{eq:initial}
v x_2 + Q(x_1, y_1, x_2, y_2) = x_2 (\mu x_2^2 + \nu y_2^2),
\end{equation}
где $\mu$, $\nu$ "--- вещественные параметры, одновременно не равные нулю. Такая постановка задачи возникла в связи с тем, что в данном классе уже имеются примеры однородных поверхностей~\cite{ALS}:
\begin{equation}\label{eq:homogenous}
v = \frac{x_1^2}{1 - x_2} + |z_2|^2,\ (x_2 \ne 1).
\end{equation}
Однако, класс поверхностей~(\ref{eq:initial}) не изучался целенаправленно и требует дальнейшего рассмотрения.

В современной науке сформировалось несколько подходов к изучению симметрий поверхностей "--- геометрический, аналитический и алгебраический. Первый подход изучает симметрии поверхностей с помощью дифференциальной геометрии и изучения различных инвариантных структур на многообразиях. Яркими примерами работ в этом направлении являются книги геометров бельгийской дифференциально-геометрической школы Ф. Диллена и Л. Вранкена (см. \cite{dillen}, \cite{vrancken}). К сожалению, этот метод, хотя и является естественным, не дал весомых результатов (к примеру, задача построения полного списка однородных поверхностей пространства $\mathbb{R}^3$ так и не была решена).

Второй подход связан с построением канонических (относительно заданного класса преобразований) уравнений и был предложен А. В. Лободой. Вопросы существования и построения нормальных форм для невырожденных по Леви вещественных многообразий рассматриваются в работах~\cite{loboda_hodarev} и~\cite{danilov}. Следует заметить, что именно этот метод позволил существенно продвинуться в классификации однородных многообразий.

В данной работе используется третий (алгебраический) метод, основанный на изучении симметрий исходных объектов с помощью групп Ли и инфинитезимальных преобразований. Подробное описание метода определения размерности групп преобразований, сохраняющих поверхности из изучаемого класса, представлено в следующей главе данной работы.

\end{document}