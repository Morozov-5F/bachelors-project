% !TEX root = ../main.tex
\documentclass[../main.tex]{subfiles}
\begin{document}

В данной работе рассматривается задача, связанная с описанием симметрий вещественных гиперповерхностей многомерных комплексных пространств.

Симметричные множества представляли интерес в математике со времен зарождения этой науки. Для подтверждения этого, достаточно вспомнить, что в геометрии Евклида важное место занимали правильные треугольники и четырехугольники (квадраты), а так же окружности.

И квадрат, и окружность являются симметричными множествами. Однако, окружность <<одинаково устроена>> в каждой своей своей точке, в то время как квадрат имеет вершины и точки, которые не являются симметричными. Это свойство множества называется в современной математике однородностью и является обобщением более простого свойства точечной симметрии.

Помимо тел на плоскости, в ранней геометрии также изучались симметричные тела в трехмерном пространстве. Так, уже древним грекам были известны описания всех правильных многогранников "--- <<платоновых тел>> "--- тетраэдра, октаэдра, гексаэдра (куба), икосаэдра и додекаэдра. Также был известен тот факт, что список этих тел полон и что не существует, к примеру, правильного 115-гранника. Однако, строгое доказательство этого факта не является тривиальным и требует определенной математической сноровки. По аналогии с планиметрией, в стереометрии при рассмотрении платоновых тел сфера упоминается как <<правильный многогранник с бесконечным числом граней>>, обладающий из-за этого бесконечным количеством симметрий. Сфера также является однородным объектом с многих <<естественных>> точек зрения.

Во второй половине XIX века различные симметрии математических объектов стали изучаться на качественно новом уровне строгости. Связано это было с изучением теории групп и, в частности, с развитием теории непрерывных групп. В связи с этим можно упомянуть работы Ф. Клейна, и, например его труд <<Лекции об икосаэдре и решении уравнений пятой степени>>.

Непрерывные группы (преобразований) получили наибольшую завершенность в работах норвежского математика Софуса Ли, например, в его фундаментальном труде <<Теория групп преобразований>>~\cite{lie}. Непрерывные группы преобразований в идеальной их форме получили название <<группы Ли>> и именно в терминах этих групп производятся современные исследования симметрий различных математических объектов.

Начиная с середины XX века, активно начинают исследоваться симметрии различных математических объектов. Тут следует упомянуть труд У. Миллера <<Группы симметрий и их приложения>>, в котором кратко изложен математический аппарат для изучения симметрий с точки зрения непрерывных групп~\cite{miller1973symmetry}.

Из современных исследований, которые касаются симметрий алгебраических объектов следует выделить работы А. Исаева и Б. Кругликова, к примеру, <<On the Symmetry Algebras of 5-dimensional CR-manifolds>>, в которой изучаются симметрии вещественных гиперповерхностей в трехмерном комплексном пространстве~\cite{IK}. Этот труд очень близок по тематике к данной работе.

В современной математике представляет интерес задачи, связанные с поиском и описанием различных однородных объектов в вещественных и комплексных пространствах. Сходные по своей сути задачи ставились и решались математиками уже в конце XIX "--- начале XX веков. Однородные относительно аффинных преобразований поверхности 3-мерного вещественного пространства были описаны в середине XX-го века, и тогда же эти описания были включены в учебники по дифференциальной геометрии~\cite{shirokov}.

В 1932 г. Э. Картан построил полный список вещественных гиперповерхностей 2-мерных комплексных пространств, которые являются однородными относительно голоморфных преобразований~\cite{cartan}. Решение же более простой по постановке задачи описания аффинно-однородных гиперповерхностей пространства $\mathbb{C}^2$ было получено А.В. Лободой сравнительно недавно~\cite{loboda_c2}.

Задачу описания свойства аффинной однородности можно рассматривать как часть большей проблемы, которая связана с изучением голоморфной однородности. Близость этих задач обусловлена тем, что в рамках второй задачи естественно выделяется класс аффинно-однородных объектов, которые являются однородными относительно голоморфных преобразований.

В настоящее время имеется большое число частных результатов об однородных многообразиях в пространстве $\mathbb{C}^3$ (\cite{ALS},~\cite{loboda_hodarev}). Однако, задача, в силу своей объемности и сложности, остается нерешенной.

%% Конкретное уравнение (Архив) настойчивый интерес к кубичским поверхностям
$$
v = x_1^2 / x_2 + x_2^{\alpha}
$$
%% Иствуд - Ежов (поверхность кэли)
$$
x_3 = x_1 x_2 + x_1^3
$$

В данной работе изучаются симметрии одного семейства кубических вещественных гиперповерхностей в пространстве $\mathbb{C}^3$. Главной задачей является исследование групп аффинных преобразований, сохраняющих каждую поверхность из этого семейства и их размерностей.

Семейство поверхностей, рассматриваемое в данной задаче, является одним из объектов исследования в связи с указанной задачей об аффинной однородности вещественных многообразий в трехмерном комплексном пространстве.

В первых главах этой работы производится постановка задачи на формальном уровне, а так же вводятся основные понятия и методика, необходимая для решения поставленной задачи.

Основным результатом данной работы является оценка размерности группы аффинных преобразований, сохраняющих все поверхности из исследуемого семейства. Соответствующая теорема и её доказательство представлены в третьем разделе; в нём же присутствует детальный разбор одного из частных случаев поверхностей из этого семейства.

В главе четыре приведено описание отдельных элементов алгоритма по нахождению искомых размерностей групп преобразований. Процедуры были реализованы при помощи пакета символьной математики \verb|Wolfram Mathematica|.

В приложениях можно найти некоторые выражения, которые слишком велики для отобрадения в тексте, но, тем не менее, представляют интерес. Также в приложение вынесен полный листинг программы на языке \verb|Wolfram Language|.

\end{document}