% !TEX root = ../main.tex
\documentclass[../main.tex]{subfiles}

\begin{document}

\newpage
\section*{\centering Реферат}

\begin{flushleft}
Бакалаврская работа \zpageref{LastPage}~с., \total{citnum}\ источников, \total{totalappendices} приложения.\\
\vspace{0.5cm}

АФФИННОЕ ПРЕОБРАЗОВАНИЕ, ГРУППЫ ЛИ, КОМПЛЕКСНОЕ ПРОСТРАНСТВО, ОДНОРОДНОЕ ПОДМНОГООБРАЗИЕ, ИНФИНИТЕЗИМАЛЬНОЕ ПРЕОБРАЗОВАНИЕ, СИСТЕМЫ УРАВНЕНИЙ, СИММЕТРИИ.

\vspace{0.5cm}
Объект исследования "--- вещественные гиперповерхности в пространстве~$\mathbb{C}^3$\\
\vspace{0.5cm}
Цель работы "--- исследование симметрий одного из классов вещественных кубических гиперповерхностей в трехмерном комплексном пространстве.\\
\vspace{0.5cm}
Метод исследования и аппаратура "--- теория групп преобразований, многомерный комплексный анализ, теория алгоритмов, математический пакет символьных вычислений {\ttfamily Wolfram Mathematica}, персональный компьютер.\\
\vspace{0.5cm}

В настоящей работе было изучены размерности групп аффинных преобразований, сохраняющих поверхности из данного семейства. В ходе исследования была разработана программа на языке~{\ttfamily Wolfram Language} для решения поставленной задачи. Тестирование алгоритмов и проверка результатов были осуществлены с помощью пакета символьных вычислений {\ttfamily Wolfram Mathematica}.

\end{flushleft}
\clearpage
\normalsize

\end{document}