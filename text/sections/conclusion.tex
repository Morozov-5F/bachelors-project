% !TEX root = ../main.tex
\documentclass[../main.tex]{subfiles}
\begin{document}
В представленной работе исследованы симметрии одного класса кубических гиперповерхностей в трехмерном комплексном пространстве. Такая постановка задачи тесно связана с проблемой описания однородных подмножеств многомерных пространств и с построением относительно простых примеров  однородных объектов.  

В работе приведены теоретические рассуждения, позволяющие перейти от формальных описаний групп и алгебр Ли, связанных с обсуждаемыми поверхностями, к конкретным вычислительным процедурам.

Большое количество параметров, возникающих в изучаемой задаче, потребовало привлечения  компьютерных вычислений в пакете символьной математики {\ttfamily Wolfram Mathematica}. В работе  имеются  приложения в виде распечаток отдельных фрагментов вычислительных программ. 

Показано, что даже при использовании современных высокотехнологичных информационных систем и, в частности, пакетов символьных вычислений, полное решение изучаемой задачи невозможно без участия человеческого (естественного) интеллекта. Получены оценки для размерностей групп аффинных преобразований, сохраняющих поверхности из изучаемого семейства.  Приведены примеры поверхностей, обладающих группами всех допустимых размерностей.  Получено обобщение известных примеров аффинно-однородных поверхностей в $\mathbb{C}^3$~\cite{ALS}.

\end{document}