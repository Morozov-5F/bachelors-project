% XeLaTeX can use any Mac OS X font. See the setromanfont command below.
% Input to XeLaTeX is full Unicode, so Unicode characters can be typed directly into the source.

% The next lines tell TeXShop to typeset with xelatex, and to open and save the source with Unicode encoding.

%!TEX TS-program = xelatex
%!TEX encoding = UTF-8 Unicode
\documentclass[a4paper,14pt]{extarticle}
\usepackage{geometry}
\geometry{a4paper}                   % ... or a4paper or a5paper or ...
%\geometry{landscape}                % Activate for for rotated page geometry
\setlength{\parindent}{15mm}
%\renewcommand{\baselinestretch}{1.5}
\usepackage{graphicx}
\usepackage{amsmath, amssymb, amsthm, mathtools}
\newtheorem{proposal}{Предложение}
\newtheorem{theorem}{Теорема}
\newtheorem{lemma}{Лемма}
\newtheorem{theorem_alt}{Теорема (альтернативная формулировка)}
\newtheorem{consequence}{Следствие}
\usepackage{fontspec,xltxtra,xunicode}
\usepackage{csquotes}
\usepackage{tocloft}
\renewcommand{\cftsecleader}{\cftdotfill{\cftdotsep}}
%\defaultfontfeatures{Ligatures=TeX}
\makeatletter
\newcommand\thefontsize[1]{{#1 The current font size is: \f@size pt\par}}
\makeatother
\usepackage[titletoc, title]{appendix}
\renewcommand{\appendixtocname}{Приложения}
\renewcommand{\appendixname}{Приложение}

\defaultfontfeatures{Mapping=tex-text}
\setmainfont{Times New Roman}
%\setmainfont{CMU Serif}
\newfontfamily{\cyrillicfontsf}{Arial}
\setsansfont{Helvetica}
\newfontfamily{\cyrillicfonttt}{Courier New}
\setmonofont{Courier New}
%\setmonofont{CMU Typewriter Text}
\usepackage{newtxmath}

\usepackage{polyglossia}
\setmainlanguage{russian}
\setotherlanguage{english}
\setkeys{russian}{babelshorthands=true}

\newcommand{\sectionbreak}{\clearpage}
\usepackage{titlesec}
\usepackage{indentfirst}
\usepackage{setspace}
\onehalfspacing
\usepackage{titlesec}

\titleformat{\section}[block]{\normalsize\bfseries\centering}{\thesection}{1ex}{}
\titleformat{\subsection}[block]{\normalsize\bfseries}{\thesubsection}{1ex}{}

\usepackage{listings} 
\lstset{ 
language=[5.2]Mathematica, 
morekeywords={Solve, Evaluate, Function, Print}, 
breaklines=true, 
extendedchars=\true, 
inputencoding=utf8, 
tabsize=2, 
escapeinside={\%}{\%},
basicstyle=\small\ttfamily,
keywordstyle=\textbf,
texcl=true,
}
\addto\captionsrussian{\def\refname{Список использованных источников}}
\renewcommand{\thepage}{{\normalsize\textrm{\upshape\arabic{page}}}}
\usepackage{csquotes}
\numberwithin{equation}{section}
\usepackage{subfiles}

\newcommand{\RN}[1]{%
  \textup{\uppercase\expandafter{\romannumeral#1}}%
}

\makeatletter 
\renewcommand\@biblabel[1]{#1} 
\makeatother

\usepackage[lastpage,user]{zref}
\usepackage{totcount}
\newtotcounter{citnum} %From the package documentation
\def\oldbibitem{} \let\oldbibitem=\bibitem
\def\bibitem{\stepcounter{citnum}\oldbibitem}

\usepackage{tocloft}

\cftsetindents{section}{0em}{2em}
\cftsetindents{subsection}{0em}{2em}

\renewcommand\cfttoctitlefont{\hfill\normalsize\bfseries}
\renewcommand\cftaftertoctitle{\hfill\mbox{}}

\usepackage{array}
\newcolumntype{P}[1]{>{\centering\arraybackslash}p{#1}}

\newtotcounter{totalappendices}

\begin{document}

\newgeometry{left=3cm, right=1cm, top=2cm, bottom=2cm}

\begin{titlepage}
\subfile{sections/titlepage}
\end{titlepage}
\newgeometry{left=2cm, right=1cm, top=2cm, bottom=2cm}
%% ЗАДАНИЕ НА ВЫПОЛНЕНИЕ
\newpage\thispagestyle{empty}
\addtocounter{page}{1}
\subfile{sections/useless_page}

%% РЕФЕРАТ
\newgeometry{top=15mm, left=30mm, right=15mm, bottom=20mm}
\subfile{sections/abstract}
\tableofcontents
%\maketitle

%% ВВЕДЕНИЕ
\addcontentsline{toc}{section}{Введение}
\section*{\centering Введение}
\subfile{sections/introduction}

%%% ПОСТАНОВКА ЗАДАЧИ
\section{Постановка задачи}
\subfile{sections/first_chapter}

%%% ТЕОРЕТИЧЕСКАЯ ЧАСТЬ
\section{Симметрии в трехмерном комплексном пространстве}
\subfile{sections/second_chapter}

%%% ЭКСПЕРИМЕНТАЛЬНАЯ ЧАСТЬ
\section{Определение размерности групп аффинных преобразований}
\subfile{sections/third_chapter}

%%% КОМПЬЮТЕРНЫЕ АЛГОРИТМЫ
\section{Компьютерные алгоритмы}
\subfile{sections/fourth_chapter}

%%% ЗАКЛЮЧЕНИЕ
\section*{\centering Заключение}
\addcontentsline{toc}{section}{Заключение}
\subfile{sections/conclusion}

%%% СПИСОК ЛИТЕРАТУРЫ
\newpage
\addcontentsline{toc}{section}{Список использованных источников}
\bibliography{../bib_files/bibliography}
\bibliographystyle{ugost2003}

%%% ПРИЛОЖЕНИЕ
\begin{appendices}
\renewcommand{\thesection}{\Asbuk{section}}
\titleformat{\section}[display]
    {\normalfont\bfseries}{\addtocounter{totalappendices}{1}\appendixname\enspace\centering\thesection}{.5em}{}
\addtocontents{toc}{\protect\setcounter{tocdepth}{1}}

\section{Используемые выражения}
\subfile{sections/appendix_one}

\section{Листинг программы}
\subfile{sections/appendix_two}

\section{Схема матрицы в частном случае}
\subfile{sections/appendix_three}

\section{Схема матрицы для расширенного класса}
\subfile{sections/appendix_four}

\end{appendices}

\end{document}