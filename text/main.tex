% XeLaTeX can use any Mac OS X font. See the setromanfont command below.
% Input to XeLaTeX is full Unicode, so Unicode characters can be typed directly into the source.

% The next lines tell TeXShop to typeset with xelatex, and to open and save the source with Unicode encoding.

%!TEX TS-program = xelatex
%!TEX encoding = UTF-8 Unicode
%!BIB program = biber

\documentclass[a4paper,14pt]{extarticle}
\usepackage{geometry}
\geometry{a4paper}                   % ... or a4paper or a5paper or ... 
%\geometry{landscape}                % Activate for for rotated page geometry
\setlength{\parindent}{15mm}
%\renewcommand{\baselinestretch}{1.5}
%\usepackage[parfill]{parskip}    % Activate to begin paragraphs with an empty line rather than an indent
\usepackage{graphicx}
\usepackage{amsmath, amssymb, amsthm}
\newtheorem{proposal}{Предложение}
\newtheorem{theorem}{Теорема}
\newtheorem{lemma}{Лемма}
\newtheorem{theorem_alt}{Теорема (альтернативная формулировка)}
\newtheorem{consequence}{Следствие}
%\usepackage{mathptmx}
%\usepackage[utf8]{inputenc}
\usepackage{fontspec,,xltxtra,xunicode}
\usepackage{csquotes}
\usepackage{tocloft}
\renewcommand{\cftsecleader}{\cftdotfill{\cftdotsep}}
%\defaultfontfeatures{Ligatures=TeX}

\defaultfontfeatures{Mapping=tex-text}
\setmainfont{CMU Serif}
\setsansfont{CMU Sans Serif}
\setmonofont{CMU Typewriter Text}

\usepackage[english, russian]{babel}
\newcommand{\sectionbreak}{\clearpage}
\usepackage{titlesec}
\usepackage{indentfirst}
\usepackage{setspace}
\onehalfspacing
\usepackage{titlesec}
\titleformat*{\section}{\Large\bfseries\centering}
\titleformat*{\subsection}{\large\bfseries\centering}

\usepackage[
	style=gost-numeric,
	language=auto,
	autolang=other,
	sorting=none,
	backend=biber,
	bibencoding=utf8,
	isbn=false,
	sortcites=true,
	clearlang=true,
	language=autobib,
	defernumbers=true,
]{biblatex}
\addbibresource{../bib_files/bibliography.bib}

% Will Robertson's fontspec.sty can be used to simplify font choices.
% To experiment, open /Applications/Font Book to examine the fonts provided on Mac OS X,
% and change "Hoefler Text" to any of these choices.
%\setromanfont[Mapping=tex-text]{Times New Roman}
%\setsansfont[Scale=MatchLowercase,Mapping=tex-text]{Gill Sans}
%\setmonofont[Scale=MatchLowercase]{Courier New}

%\title{Brief Article}
%\author{The Author}
\date{}                                           % Activate to display a given date or no date

\begin{document}
\begin{titlepage}

\thispagestyle{empty}
\newgeometry{margin=2cm}
\center 
\textsc{МИНОБРНАУКИ РОССИИ}\\
\textsc{ФЕДЕРАЛЬНОЕ ГОСУДАРСТВЕННОЕ БЮДЖЕТНОЕ ОБРАЗОВАТЕЛЬНОЕ УЧРЕЖДЕНИЕ ВЫСШЕГО ОБРАЗОВАНИЯ <<ВОРОНЕЖСКИЙ ГОСУДАРСТВЕННЫЙ УНИВЕРСИТЕТ>>}

\vspace{0.3cm}

\textrm{Факультет компьютерных наук}\\
\textrm{Кафедра цифровых технологий}

\vspace{1cm}

\textbf{Исследование симметрий алгебраических уравнений}\\

\vspace{1cm}

\textrm{ВКР Бакалаврская работа}\\
\textrm{02.03.01 Математика и компьютерные науки}\\
\textrm{Распределенные системы и искусственный интеллект}\\

\vfill
\begin{flushleft}
\textrm{Допущено к защите в ГЭК \rule[0mm]{5mm}{0,3mm} . \rule[0mm]{2cm}{0,3mm} . 2017 г.}
\end{flushleft}
\begin{tabbing}
ооооооооооооооо	\=	----------------------	\kill
Зав. кафедрой	\> 	\rule[0mm]{5cm}{0,3mm}	\textit{С.Д. Кургалин, д. ф.-м. н., профессор}  \\
Обучающийся 	\> 	\rule[0mm]{5cm}{0,3mm}	\textit{Е.Ю. Морозов, 4 курс, д/о} \\ 
ооооооооооооооо	\=	----------------------	\kill
Руководитель	\> 	\rule[0mm]{5cm}{0,3mm}  \textit{А.В. Лобода, д. ф.-м. н., профессор}
\end{tabbing}

\vfill

\centerline{Воронеж, 2017}
\clearpage
\end{titlepage}
%% ЗАДАНИЕ НА ВЫПОЛНЕНИЕ
\newpage\thispagestyle{empty}
\addtocounter{page}{1}
\newgeometry{margin=2cm}
\begin{spacing}{1.2}
{
\begin{center}
{\small МИНОБРНАУКИ РОССИИ}\\  \!  \!  \! 
{\small \textbf{ФЕДЕРАЛЬНОЕ ГОСУДАРСТВЕННОЕ БЮДЖЕТНОЕ}}\\ \!  \!  \!  
{\small \textbf{ОБРАЗОВАТЕЛЬНОЕ УЧРЕЖДЕНИЕ}}\\ \!  \! 
{\small  \textbf{ВЫСШЕГО ОБРАЗОВАНИЯ}\\ \!  \! 
\textbf{''ВОРОНЕЖСКИЙ ГОСУДАРСТВЕННЫЙ УНИВЕРСИТЕТ''}\\ \!  \! 
Факультет компьютерных наук\\  \!  \! \!
$~$Кафедра цифровых технологий}\\
\vspace{0.1cm}
\end{center}
\begin{flushright} \!  \!  \! \! 
{\small \textbf{УТВЕРЖДАЮ}\\
заведующий кафедрой\\
 $\underset{\text{\emph{подпись}}}{\underline{\hspace{0.15\textwidth}}}$ $\underset{\text{\emph{расшифровка подписи}}}{\underline{\hspace{0.3\textwidth}}}$}
\end{flushright}
\begin{center}
{\small \textbf{ЗАДАНИЕ \\
НА ВЫПОЛНЕНИЕ ВЫПУСКНОЙ КВАЛИФИКАЦИОННОЙ РАБОТЫ\\
ОБУЧАЮЩЕГОСЯ} $\underset{\text{\emph{фамилия, имя, отчество}}}{\underline{\hspace{0.69\textwidth}}}$}
\end{center}\! \! \! 
\vspace{0.1cm}
{\small 1. Тема работы \underline{\phantom{aaaaaaaaaaaaaaaaaaaaaaaaaaaaaaaaaaaaaaaaaaaaaa}}, утверждена решением ученого совета \underline{\phantom{aaaaaaaaaaaaaaaaaaaaaaa}} факультета от \underline{\phantom{aaa}}.\underline{\phantom{aaa}}.20\underline{\phantom{aaa}}\\ 
2. { Направление подготовки $\underset{\text{\emph{шифр, наименование}}}{\underline{\hspace{0.67\textwidth}}}$\\
3. Срок сдачи студентом законченной работы \underline{\phantom{aaa}}.\underline{\phantom{aaa}}.20\underline{\phantom{aaa}}\\
4. Календарный план:}\\  
\begin{tabular}[t]{|p{3em}|p{20em}|p{13em}|}
\hline
{\small	№} & {\small	Задание} & {\small Срок выполнения }\\
\hline
	{\small	$1$} & {} & {} \\
\hline
	{\small	$2$} & {} & {} \\
\hline 
	{\small	$3$} &{} & {} \\
\hline
	{\small	$4$} &{} & {} \\
\hline
	{\small	$5$} &{} & {} \\
\hline
	{\small	$6$} &{} & {} \\
\hline
\end{tabular}\! \! \! \!
\begin{flushleft}
\vspace{0.4cm}
{\small
Обучающийся $~~~~~~~~\underset{\text{\emph{подпись}}}{\underline{\hspace{0.2\textwidth}}}$ $\underset{\text{\emph{расшифровка подписи}}}{\underline{\hspace{0.4\textwidth}}}$\\
\vspace{0.4cm}
Руководитель $~~~~~~~~\underset{\text{\emph{подпись}}}{\underline{\hspace{0.2\textwidth}}}$ $\underset{\text{\emph{расшифровка подписи}}}{\underline{\hspace{0.4\textwidth}}}$}
\end{flushleft}\! \! \! \! \! \! \! \!
\begin{flushleft}
\underline{\phantom{a}} \underline{\phantom{a}} \underline{\phantom{a}} \underline{\phantom{a}} \underline{\phantom{a}} \underline{\phantom{a}} \underline{\phantom{a}} \underline{\phantom{a}} \underline{\phantom{a}} \underline{\phantom{a}} \underline{\phantom{a}} \underline{\phantom{a}} \underline{\phantom{a}} \underline{\phantom{a}} \underline{\phantom{a}} \underline{\phantom{a}} \underline{\phantom{a}} \underline{\phantom{a}} \underline{\phantom{a}} \underline{\phantom{a}} \underline{\phantom{a}} \underline{\phantom{a}} \underline{\phantom{a}} \underline{\phantom{a}} \underline{\phantom{a}} \underline{\phantom{a}} \underline{\phantom{a}} \underline{\phantom{a}} \underline{\phantom{a}} \underline{\phantom{a}}
\underline{\phantom{a}} \underline{\phantom{a}} \underline{\phantom{a}} \underline{\phantom{a}} \underline{\phantom{a}} \underline{\phantom{a}} \underline{\phantom{a}} \underline{\phantom{a}} \underline{\phantom{a}} \underline{\phantom{a}} \underline{\phantom{a}} \underline{\phantom{a}} \underline{\phantom{a}} \underline{\phantom{a}} \underline{\phantom{a}} \underline{\phantom{a}} \underline{\phantom{a}} \underline{\phantom{a}} \underline{\phantom{a}} 
\end{flushleft}\! \! \! \!
{\small
Выпускная квалификационная работа представлена на кафедру \underline{\phantom{aaa}}.\underline{\phantom{aaa}}.20\underline{\phantom{aaa}}\\
Рецензент $\underset{\text{\emph{должность, ученая степень, ученое звание}}}{\underline{\hspace{0.9\textwidth}}}$\\
Выпускная квалификационная работа на тему {\underline{\hspace{0.5\textwidth}}}\\
{\underline{\hspace{1\textwidth}}}\\
допущена к защите в ГЭК \underline{\phantom{aaa}}.\underline{\phantom{aaa}}.20\underline{\phantom{aaa}}\\
Заведующий кафедрой $\underset{\text{\emph{подпись, расшифровка подписи}}}{\underline{\hspace{0.4\textwidth}}}$  \underline{\phantom{aaa}}.\underline{\phantom{aaa}}.20\underline{\phantom{aaa}}\\
}
}}
\end{spacing}

%% РЕФЕРАТ
\newpage\thispagestyle{empty}
\begin{center}
РЕФЕРАТ
\end{center}

\begin{flushleft}
Бакалаврская работа  с.,  источника,  приложения\\
\vspace{0.5cm}

\vspace{0.5cm}
Объект исследования "--- \\
\vspace{0.5cm}
Цель работы "--- \\
\vspace{0.5cm}
Метод исследования и аппаратура "---\\
\vspace{0.5cm}

Полученные результаты и их новизна "---\\
\vspace{0.5cm}
Область применения "---\\
\vspace{0.5cm}
Прогнозные предположения о развитии объекта исследования "---.
\end{flushleft}

\normalsize
\newgeometry{top=15mm, left=30mm, right=15mm, bottom=20mm}
\tableofcontents
%\maketitle

% For many users, the previous commands will be enough.
% If you want to directly input Unicode, add an Input Menu or Keyboard to the menu bar 
% using the International Panel in System Preferences.
% Unicode must be typeset using a font containing the appropriate characters.
% Remove the comment signs below for examples.

% \newfontfamily{\A}{Geeza Pro}
% \newfontfamily{\H}[Scale=0.9]{Lucida Grande}
% \newfontfamily{\J}[Scale=0.85]{Osaka}

% Here are some multilingual Unicode fonts: this is Arabic text: {\A السلام عليكم}, this is Hebrew: {\H שלום}, 
% and here's some Japanese: {\J 今日は}.

%% ВВЕДЕНИЕ
\addcontentsline{toc}{section}{Введение}
\section*{\centering Введение}
В данной работе рассматривается задача, связанная с описанием симметрий вещественных гиперповерхностей многомерных комплексных пространств.

Симметричные множества представляли интерес в математике со времен зарождения этой науки. Для подтверждения этого, достаточно вспомнить, что в геометрии Евклида важное место занимали правильные треугольники и четырехугольники (квадраты), а так же окружности. 

И квадрат, и окружность являются симметричными множествами. Однако, окружность <<одинаково устроена>> в каждой своей своей точке, в то время как квадрат имеет вершины и точки, которые не являются симметричными. Это свойство множества называется в современной математике однородностью и является обобщением более простого свойства точечной симметрии.

Помимо тел на плоскости, в ранней геометрии также изучались симметричные тела в трехмерном пространстве. Так, уже древним грекам были известны описания всех правильных многогранников "--- <<платоновых тел>> "--- тетраэдра, октаэдра, гексаэдра (куба), икосаэдра и додекаэдра. Также был известен тот факт, что список этих тел полон и что не существует, к примеру, правильного 115-гранника. Однако, строгое доказательство этого факта не является тривиальным и требует определенной математической сноровки. По аналогии с планиметрией, в стереометрии при рассмотрении платоновых тел сфера упоминается как <<правильный многогранник с бесконечным числом граней>>, обладающий из-за этого бесконечным количеством симметрий. Сфера также является однородным объектом с многих <<естественных>> точек зрения.

Во второй половине XIX века различные симметрии математических объектов стали изучаться на качественно новом уровне строгости. Связано это было с изучением теории групп и, в частности, с развитием теории непрерывных групп. В связи с этим можно упомянуть работы Ф. Клейна, и, например его труд <<Лекции об икосаэдре и решении уравнений пятой степени>>.

Непрерывные группы (преобразований) получили наибольшую завершенность в работах норвежского математика Софуса Ли, например, в его фундаментальном труде <<Теория групп преобразований>>~\cite{lie}. Непрерывные группы преобразований в идеальной их форме получили название <<группы Ли>> и именно в терминах этих групп производятся современные исследования симметрий различных математических объектов.

Начиная с середины XX века, активно начинают исследоваться симметрии различных математических объектов. Тут следует упомянуть труд У. Миллера <<Группы симметрий и их приложения>>, в котором кратко изложен математический аппарат для изучения симметрий с точки зрения непрерывных групп~\cite{miller1973symmetry}. 

Из современных исследований, которые касаются симметрий алгебраических объектов следует выделить работы А. Исаева и Б. Кругликова, к примеру, <<On the Symmetry Algebras of 5-dimensional CR-manifolds>>, в которой изучаются симметрии вещественных гиперповерхностей в трехмерном комплексном пространстве~\cite{IK}. Этот труд очень близок по тематике к данной работе.

В современной математике представляет интерес задачи, связанные с поиском и описанием различных однородных объектов в вещественных и комплексных пространствах. Сходные по своей сути задачи ставились и решались математиками уже в конце XIX "--- начале XX веков. Однородные относительно аффинных преобразований поверхности 3-мерного вещественного пространства были описаны в середине XX-го века, и тогда же эти описания были включены в учебники по дифференциальной геометрии~\cite{shirokov}.

В 1932 г. Э. Картан построил полный список вещественных гиперповерхностей 2-мерных комплексных пространств, которые являются однородными относительно голоморфных преобразований~\cite{cartan}. Решение же более простой по постановке задачи описания аффинно-однородных гиперповерхностей пространства $\mathbb{C}^2$ было получено А.В. Лободой сравнительно недавно~\cite{loboda_c2}.

Задачу описания свойства аффинной однородности можно рассматривать как часть большей проблемы, которая связана с изучением голоморфной однородности. Близость этих задач обусловлена тем, что в рамках второй задачи естественно выделяется класс аффинно-однородных объектов, которые являются однородными относительно голоморфных преобразований.

В настоящее время имеется большое число частных результатов об однородных многообразиях в пространстве $\mathbb{C}^3$ (\cite{ALS},~\cite{loboda_hodarev}). Однако, задача, в силу своей объемности и сложности, остается нерешенной. 

В данной работе изучаются симметрии одного семейства квадро-кубических вещественных гиперповерхностей в пространстве $\mathbb{C}^3$. Главной задачей является исследование групп аффинных преобразований, сохраняющих каждую поверхность из данного семейства и их размерностей. 

Семейство поверхностей, рассматриваемое в данной задаче, является одним из объектов исследования в связи с указанной задачей об аффинной однородности вещественных многообразий в трехмерном комплексном пространстве.

В первых главах этой работы производится постановка задачи на формальном уровне, а так же вводятся основные понятия и методика, необходимая для решения поставленной задачи.

Основным результатом данной работы является оценка размерности группы аффинных преобразований, сохраняющих все поверхности из исследуемого семейства. Соответствующая теорема и её доказательство представлены в третьем разделе; в нём же присутствует детальный разбор одного и частных случаев поверхностей из этого семейства.

В главе четыре приведено описание отдельных элементов алгоритма по нахождению искомых размерностей групп преобразований. Процедуры были реализованы при помощи пакета символьной математики \verb|Wolfram Mathematica|.

%%%
%%% ОСНОВНАЯ ЧАСТЬ
%%%

%%% ПОСТАНОВКА ЗАДАЧИ
\section{Постановка задачи}
Пусть дано семейство вещественных квадро-кубических гиперповерхностей в трехмерном комплексном пространстве:
\begin{equation}\label{eq:initial}
v x_2 = Q(x_1, y_1, x_2, y_2) + x_2 (\mu x_2^2 + \nu y_2^2),
\end{equation}
где $x_1 = \Re(z_1)$, $y_1 = \Im(z_1)$, $x_2 = \Re(z_2)$, $y_2 = \Im(z_2)$, $u = \Re(w)$, $v= \Im(w)$ "--- компоненты координат в трехмерном комплексном пространстве, $\mu$, $\nu$ "--- некоторые вещественные параметры, одновременно не равные нулю, а $Q(x_1, y_1, x_2, y_2)$ "--- некоторая квадратичная форма. 
%%% ТЕОРЕТИЧЕСКАЯ ЧАСТЬ
\section{Симметрии в трехмерном комплексном пространстве}
TODO
\subsection{Группы Ли аффинных преобразований}
TODO
\subsection{Инфинитезимальные преобразования}
TODO
\subsection{Системы полиномиальных уравнений}
TODO
 
%%% ОПИСАНИЕ ПРОГРАММЫ

%%% ЭКСПЕРИМЕНТАЛЬНАЯ ЧАСТЬ
\section{Определение размерности групп аффинных преобразований}
В данном разделе будет описана процедура определения размерностей групп аффинных преобразований для семейства поверхностей~(\ref{eq:initial}).
\subsection{Общая схема}
Пусть дана некоторая вещественная гиперповерхность $M$ из семейства~(\ref{eq:initial}) с определяющей функцией $$\Phi(x_1, x_2, y_1, y_2, v) = v x_2 - Q(x_1, y_1, x_2, y_2) - x_2 (\mu x_2^2 + \nu y_2^2).$$ В таком случае, обозначим через $F_t$ однопараметрическую группу Ли аффинных преобразований, сохраняющих эту поверхность. Пусть также эта группа содержит тождественное преобразование при $t = 0$, т.е. $F_0 = \textrm{Id}$. Тогда уравнения, определяющие эту группу будут иметь вид:
\begin{equation}
  \begin{cases}
     z_1 &=~A_{1}(t) z_1^* + A_{2}(t) z_2^* + A_{3}(t) w^* + P_1(t) \\ 
     z_1 &=~B_{1}(t) z_1^* + B_{2}(t) z_2^* + B_{3}(t) w^* + P_2(t) \\ 
     w   &=~C_{1}(t) z_1^* + C_{2}(t) z_2^* + C_{3}(t) w^* + q(t)
  \end{cases},
\end{equation}
где $A_{i}(t), B_{i}(t), C_{i}(t), P_i(t)$ и $q(t)$ "--- некоторые аналитические комплексные функции от вещественного аргумента. Аналитичность функций возможна, так как группа $F_t$ является группой Ли. Для каждой компоненты координат в комплексном пространстве справедливы следующие уравнения:
\begin{equation}
  \begin{cases}
     x_1 &=~x_1 a_{1,1} - y_1 a_{1,2} + x_2 a_{2,1} - y_2 a_{2,2} + u a_{3,1} - v a_{3,2} + p_{1,1} \\ 
     y_1 &=~x_1 a_{1,2} + x_2 a_{2,2} + y_1 a_{1,1} + y_2 a_{2,1} + u a_{3,2} + v a_{3,1} + p_{1,2} \\
     x_2 &=~x_1 b_{1,1} + x_2 b_{2,1} - y_1 b_{1,2} - y_2 b_{2,2} + u b_{3,1} - v b_{3,2} + p_{2,1}\\
     y_2 &=~x_1 b_{1,2} + x_2 b_{2,2} + y_1 b_{1,1} + y_2 b_{2,1} + u b_{3,2} + v b_{3,1} + p_{2,2}\\
     v   &=~x_1 c_{1,2} + x_2 c_{2,2} + y_1 c_{1,1} + y_2 c_{2,1} + u c_{3,2} + v c_{3,1}+q_2
  \end{cases},
\end{equation}
где $a_{i, 1} = \Re\{A_i(t)\}$, $a_{i, 2} = \Im\{A_i(t)\}$, $b_{i, 1} = \Re\{B_i(t)\}$, $b_{i, 2} = \Im\{B_i(t)\}$, $c_{i, 1} = \Re\{C_i(t)\}$, $c_{i, 2} = \Im\{C_i(t)\}$, $p_{i, 1} = \Re\{P_i(t)\}$, $p_{i, 2} = \Im\{P_i(t)\}$, $q_1 = \Re\{q(t)\}$ и $q_2 = \Im\{q(t)\}$ "--- вещественные функции от аргумента $t$ (опущен для компактности записи). 
Под действием группы $F_t$ определяющая функция примет следующий вид:
\begin{align*}
F_t(\Phi) &= 
\end{align*}

\subsection{Произвольная поверхность}
В общем случае, квадратичную форму $Q(x_1, y_1, x_2, y_2)$ можно записать в явном виде: 
\begin{align*}
Q(x_1, y_1, x_2, y_2) &= k_1 x_1^2 + k_2 x_2 x_1 + k_3 x_2^2 + k_4 x_1 y_1 + k_5 x_2 y_1 \\
&+ k_6 y_1^2 + k_7 x_1 y_2 + k_8 x_2 y_2 + k_9 y_1 y_2 + k_{10} y_2^2
\end{align*}

Применяя метод, указанный в предыдущем пункте, получим линейную однородную систему из 83 уравнений относительно 24 неизвестных, причем переменная $\sigma_{3,1}$ является <<фиктивной>> "--- в связи с тем, что   Наиболее простые уравнения в системе имеют вид
\begin{equation}
\begin{array}{l}
 k_1 \beta _{3,2}=0 \\
 k_1 \beta _{3,1}=0 \\
 k_1 \sigma _{2,1}=0 \\
 k_4 \beta _{3,1}=0 \\
 k_4 \sigma _{2,1}=0 \\
 k_7 \beta _{3,1}=0 \\
 k_7 \sigma _{2,1}=0 \\
 k_6 \beta _{3,1}=0 \\
 k_6 \sigma _{2,1}=0 \\
 k_9 \beta _{3,1}=0 \\
 k_9 \sigma _{2,1}=0 \\
 k_6 \beta _{3,2}=0 \\
 k_{10} \beta _{3,2}=0 \\
\end{array}
\end{equation}
Отсюда можно выделить два случая:
\begin{enumerate}
	\item Все $k_j$ равны нулю: $k_1 = k_4 = k_6 = k_7 = k_9 = k_{10} = 0$.
	\item Хотя бы один $k_j$ из указанного набора не равен нулю.
\end{enumerate}
В первом случае поверхность приобретает следующий вид:
\begin{equation*}
v x_2 = k_2 x_2 x_1 + k_3 x_2^2 + k_5 x_2 y_1 + k_8 x_2 y_2 + x_2 (\mu x_2^2 + \nu y_2^2).
\end{equation*}
Легко видеть, что данное уравнение можно поделить на $x_2$, что в итоге даст новую поверхность, которая является квадрикой. Рассмотрение поверхностей данного типа не представляет интереса в данной работе, поэтому можно сразу перейти к рассмотрению второго случая.

Предположим, что $k_1 \ne 0 $. В таком случае, приведем элементарными преобразованиями матрицу системы $W$ к ступенчатому виду. 

Исходя их этих соображений можно сформулировать следующую теорему: 
\begin{theorem} Размерность группы Ли аффинных преобразований, сохраняющих любую поверхность из семейства~(\ref{eq:initial}), удовлетворяет неравенству $1 \le \dim G \le 5$, и каждая из размерностей достижима.
\end{theorem}

\subsection{Частный случай семейства}
В связи со сложностью рассмотрения общего случая, был рассмотрен частный случай семейства поверхностей~(\ref{eq:initial}), в котором квадратичная форма $Q$ не зависит от переменных $x_2$ и $y_2$. В таком случае, семейство может быть описано следующим образом:
\begin{equation}\label{eq:special}
v x_2 = k_1 x_1^2 + k_2 x_1 y_1 + k_3 x_2^2 + x_2 (\mu x_2^2 + \nu y_2^2),
\end{equation}

Следует заметить, что за счёт применения невырожденных аффинных преобразований, можно фактически сократить число параметров $k_1, k_2$ и $k_3$ до одного.

В результате применения описанного метода, получается однородная линейная система из 57 уравнений относительно 24 неизвестных.

\begin{theorem} размерность группы Ли аффинных преобразований, сохраняющих любую поверхность вида~(\ref{eq:special}), удовлетворяет неравенствам
$3 \le \dim G \le 5$, причем
\begin{itemize}
	\item $\dim G$ = 3 достигается на поверхностях вида
	\begin{equation}\label{eq:special_3}
		v x_2 = k x_1^2 + y_1^2 + x_2 (\mu x_2^2 + \nu y_2^2),\ k \ne 0, \ k \ne 1;
	\end{equation}
	\item $\dim G$ = 4 достигается на поверхностях вида
	\begin{equation}\label{eq:special_4}
		v x_2 = |z_1|^2 + x_2 (\mu x_2^2 + \nu y_2^2);
	\end{equation}
		\item $\dim G$ = 5 достигается на аффинно-однородных поверхностях следующего вида
	\begin{equation}\label{eq:special_5}
		v x_2 = x_1^2 + x_2 (\mu x_2^2 + \nu y_2^2).
	\end{equation}
\end{itemize}
\end{theorem}

\section{Компьютерные алгоритмы}
TODO
%%% ЗАКЛЮЧЕНИЕ
\addcontentsline{toc}{section}{Заключение}
\section*{\centering Заключение}
TODO

\addcontentsline{toc}{section}{Список использованных источников}
\printbibliography

\end{document}  